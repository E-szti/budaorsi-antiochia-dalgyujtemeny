\documentclass[a5paper,twoside]{article}
\usepackage[utf8]{inputenc}
\usepackage{lmodern}
\usepackage[T1]{fontenc}
\usepackage[hungarian]{babel}

\usepackage{graphicx}
\graphicspath{ {images/} }

\usepackage[bookmarks]{hyperref}

% \usepackage[chorded]{songs}
\usepackage[lyric]{songs}
% \usepackage[slides]{songs}

\title{Gitáros mise\\\textit{(2020. november 01.)}}
% \author{Donkó István}
\date{}

\setlength{\topmargin}{-1.5cm}

\setlength{\headheight}{0cm}
\setlength{\headsep}{0cm}
\setlength{\footskip}{0.85cm}

\setlength{\oddsidemargin}{-1.3cm}
\setlength{\evensidemargin}{-1.8cm}

\setlength{\textwidth}{12.9cm}
\setlength{\textheight}{18.5cm}

\setlength{\songnumwidth}{1.02cm}
\setlength{\versenumwidth}{0.5cm}
% \setlength{\cbarwidth}{0.1cm}

\catcode`_=12 % use underscore as regular character
\renewcommand{\_}[1]{\underline{#1}} % shorter command for adding melismas
\songcolumns{1}
\notenames{A}{H}{C}{D}{E}{F}{G}

\begin{document}
  \pagenumbering{arabic}

  \begin{titlepage}
    \pagenumbering{gobble}
    \setlength{\oddsidemargin}{-1.625cm}

    \vspace*{4cm}
    {\let\newpage\relax\maketitle}
  \end{titlepage}

  \versesep=12pt plus 3pt minus 5pt

  \iflyric
    \baselineadj=2pt plus 1pt minus 1pt
  \fi

  % \includeonlysongs{M11,101,56,M11,101,56}
  % \includeonlysongs{80,M10,55,26,M30,M40,85,40,15}
  % \includeonlysongs{46,M10,M20,84,20,56,M30,M40,87,48,97,73,40,92}
  \includeonlysongs{6,M10,T11,M20,T101,15,M30,M41,89,85,40}
  % \includeonlysongs{T101}
  % \includeonlysongs{999}
  % \includeonlysongs{999,M41}
  % \includeonlysongs{54,90,36,80,91,77,97,40,84,11,6}

  \begin{songs}{}
    \beginsong{Add tovább}
[
  index={Add tovább}
]

  \ifchorded
    \beginverse*
      {\nolyrics Előjáték: \[D] \[F#m] \[G] \[A] \[A7]}
    \endverse
  \fi

  \beginverse\memorize
    Nagy \[D]tűz, hatalmas \[F#m]láng, csak egy \[G]szikra az, amitől \[A]kigyullad. \echo{amitől \[A7]kigyullad}
    \[D]Melegét szórja \[F#m]rád, csak a \[G]kezed kell, hogy o\[A]danyújtsad. \echo{hogy o\[A7]danyújtsad}
  \endverse

  \beginchorus
    \[G]Épp így az Isten \[D]szeretetét, \[Em] ha egyszer el\[A]fogad\[D]tad,
    Már \[G]nem kí\[D]vánsz \[G]semmi \[D]mást, csak \[G]azt, hogy \[A]add to\[D]vább.	
  \endchorus

  \beginverse
    Az ^első napsu^gár, mi a ^dermedt tavaszi ^földre száll. \echo{tavaszi ^földre száll}
    ^Melegét szórja ^rád, és a ^fák rügyeiket ^kibontják. \echo{^kibontják}
  \endverse

  \beginverse*
    Refrén
  \endverse

  \beginverse
    ^Szívemben boldog^ság, mert az ^Úr, az Isten az ^én Atyám. \echo{az ^én Atyám}
    ^Lelkemben ég a ^vágy, hogy ^megtegyem az Ő ^akaratát. \echo{az Ő ^akaratát}
  \endverse

  \beginverse*
    /: A \[G]hegy csúcsáról \[D]kiáltok: Uram! \[Em]Tárd fel a \[A]vilá\[D]gom!
    Hogy \[G]tovább \[Em]adjam \[A]én a \[G]szeretet \[A]Iste\[D]nét. :/ \rep{2}
  \endverse

\endsong
\beginsong{Adjunk hálát}[]

  \ifchorded
    \versesep=12pt plus 3pt minus 8pt
  \fi

  \ifchorded
    \beginverse*
      {\nolyrics Előjáték: \[G] \[C] \[G]}
    \endverse
  \fi

  \beginverse\memorize
    Adjunk \[C]h{\_{á}}\[G]lát az \[C]{\_{Ú}}r\[G]nak,
    Aki \[Em]meghalt \[Am]és fel\[D]támadt,
    Ő a \[C]mennybe \[G]ment és a \[C]trónon \[G]ül,
    Onnan \[Em]várjuk, \[Am]míg el\[D]jő.
  \endverse

  \beginchorus
    Allelu\[Hm]ja, az Úr az \[Em]Isten!
    Allelu\[Am]ja, a Megvál\[D]tó!
    Allelu\[Hm]ja, az Úr az \[Em]Isten!
    Zengj alle\[Am]luját! \[D7]{\_{Á}}\[G]men! \[C] \[G]
  \endchorus

  \beginverse
    Az Ő ^t{\_{e}}s^te és ^v{\_{é}}^re
    Táplál ^minket ^itt a ^Földön,
    Gyere ^és ö^rülj, velünk ^ünne^pelj,
    Tiszta ^szívvel ^áldjuk ^Őt!
  \endverse

  \beginverse*
    Refrén
  \endverse

  \beginverse
    Aki ^benne ^hisz, sose ^szomja^zik,
    Ő az ^élő ^víz fo^rrása.
    Szent^l{\_{é}}lek, ^jöjj és ^szállj le ^ránk,
    Tölts be ^minket, ^Szeretet! ^
  \endverse

  \beginverse*
    Refrén
  \endverse

\endsong

\beginsong{Áldásoddal megyünk}[]

  \beginverse
    \[D]Áldásoddal \[G]megyünk, \[Em]megyünk innen \[A]el,
    \[D]Néked éne\[G]kelünk \[Em]boldog éne\[A]ket.
  \endverse

  \beginchorus
    \[F#m]Te vagy mindig \[G]velünk, \[F#m]ha útra \[G]kelünk,
    \[F#m]Őrizd éle\[G]tünk minden \[A]nap! \[A7]
  \endchorus

  \beginverse
    ^Minden nap di^csérünk ^Téged, jó U^ram.
    ^Néked éne^kelünk ^vígan, boldo^gan.
  \endverse

  \beginchorus
    \[F#m]Maradj mindig \[G]velünk, \[F#m]ha útra \[G]kelünk,
    \[F#m]Őrizd éle\[G]tünk minden \[A]nap! \[A7]
  \endchorus

  \beginverse*
    Minden \[D]nap!
  \endverse

\endsong

\beginsong{Áldjad, én lelkem, az Urat}[]

  \ifchorded
    \beginverse*
      {\nolyrics Előjáték: \[Am] \[Em] \[Am] \[Em] \[Am]}
    \endverse
  \fi

  \beginverse\memorize
    \[Am]Áldjad, én \[Em]lelkem, az U\[Am]rat! \[Em]
    \[Am]Áld\[Em]jad, én \[C]lelkem, \[D7]az U\[E]rat! \echo{Alle\[E7]luj{\_{a}}}
    \[Am]Áldjad, én \[Em]{l\_{e}}lkem, \[F]áldjad, én \[G6]{l\_{e}}l\[C]kem,
    \[Am]Áldjad, én \[Em]lelkem, az U\[Am]rat! \[Em] \[Am] \[Em]
  \endverse

  \beginverse
    ^Zengj alle^luját az Úr^nak! ^
    ^Zengj ^alle^luját ^az Úr^nak! \echo{Alle^luj{\_{a}}}
    ^Zengj alle^{l\_{u}}ját, ^zengj alle^{l\_{u}}^ját,
    ^Zengj alle^luját az Úr^nak! ^ ^ ^
  \endverse

  \beginverse
    ^Dicsérjük ^együtt az U^rat! ^
    ^Di^csérjük ^együtt ^az U^rat! \echo{Alle^luj{\_{a}}}
    ^Dicsérjük ^{\_{e}}gyütt, ^dicsérjük ^{\_{e}}gy^ütt,
    ^Dicsérjük ^együtt az U^rat! ^ ^ ^
  \endverse

  \beginverse
    ^Áldjad, én ^bensőm, az U^rat! ^
    ^Áld^jad, én ^bensőm, ^az U^rat! \echo{Alle^luj{\_{a}}}
    ^Áldjad, én ^{b\_{e}}nsőm, ^áldjad, én ^{b\_{e}}n^sőm,
    ^Áldjad, én ^bensőm, az U^rat! ^ ^ ^
  \endverse

  \ifchorded
    \beginverse*
      {\nolyrics Utójáték: \[Am] \[Em] \[Am] \[Em] \[Am]}
    \endverse
  \fi

\endsong

\beginsong{Áldjátok az Urat}[]

  \beginverse*
    /: \[Em]Áldjátok az Ur\[Am]at, áldjátok szent ne\[Em]vét,
    Kik \[Em]házában álltok \[D]napról n\_{a}pra, \[Hm]dicsérjétek szüntelen \[Em]Őt! :/ \rep{2}
  \endverse

  \beginverse*
    /: \[Em]Tárjátok kezei\[Am]tek az él\_{ő} Isten fe\[Em]lé!
    \[Am]Áldjon meg téged \[Em]Sionból az Úr, \[D]ragyogtassa \[Hm]arcát re\[Em]ád! :/ \rep{2}
  \endverse

\endsong

\beginsong{Áldott légy, Uram}[]

  \beginverse*
    \[Dm]Áldott \[G]légy, U\[Dm]ram, szent \[A#]neved \[C]áldja \[F]lelkem! \[A]
    \[Dm]Áldott \[G]légy, U\[Dm]ram, mert \[A#]megvál\[C]tottál \[Dm]már.
  \endverse

\endsong

\beginsong{Álmaimból, lelkemből}
[
  li={Kék könyv / 12.B}
]

  \ifchorded
    \beginverse*
       {\nolyrics Előjáték:
       \[Hm] \[F#m] \[G] \[F#m] \[G] \[G] \[A] \[A]
       \[Hm] \[F#m] \[G] \[F#m] \[G] \[A] \[A] \[G] \[G] \[D] \[A]}
    \endverse
  \fi

  \beginverse\memorize
    \[D] Szürke a \[G]föld és szürke az \[D]ég, kopár az \[G]út és csillag sem \[D]ég,
    Elcsüggedt \[Em]szívű, magányos \[A]vándor, hova \[D]m\[A]{\_{é}}sz?
    \[D] Ne menj to\[G]vább, nem futhatsz \[D]el, csak rajtad \[G]áll, csak tenned \[D]kell,
    Elcsüggedt \[Em]szívű, magányos \[A]vándor, van re\[D]m\[A]{\_{é}}ny!
  \endverse

  \beginchorus
    /: \[A] Álmaim\[Hm]ból, lelkem\[F#m]ből, két kezem\[G]ből, szívem\[F#m]ből,
    Szól ez a \[G]dal, tied a \[A]dal,
    Zengjen a \[Hm]dal szíved\[F#m]ben, gyulladjon \[G]fény a lelked\[F#m]ben,
    Vezesse \[G]lépteidet az \[A]Isten, vezesse \[G]életed! \[G] \[D] :/ \rep{2}
  \endchorus

  \beginverse
    ^ Valami^kor így éltem ^én, azt hittem ^már nincs is re^mény
    Elcsüggedt ^szívű, magányos ^vándor voltam {^^{\_{é}}}n!
    ^ Valaki ^jött, rám neve^tett: Ne menj to^vább, nyújtsd a ke^zed,
    Elcsüggedt ^szívű, magányos ^vándor, sose ^f^{\_{é}}lj!
  \endverse

  \beginverse*
    Refrén \[A] \[D]
  \endverse

\endsong

\beginsong{Antióch dal}[]

  \beginverse\memorize
    \[G]Gyere, gyere, gyere és tapsoljál velünk! \echo{pop subap, pop pop subap}
    \[C]Gyere, gyere, gyere és tapsoljál velü\[G]nk! \echo{pop subap, pop pop subap}
    \[D]Gyere, gyere, gyere és énekelj, hogy \[C]jól szóljon \[C7] a dalunk:
  \endverse

  \beginchorus
    \[G] A-N-\[Em]T, A-N-T,
    \[G] A-N-\[Em]T-I-O-H,
    \[G] A-N-\[Em]T-I-\[C]O-\[D]H-I-\[G]A
  \endchorus

  \beginverse
    ^Gyere, gyere, gyere és rázd a jobb kezed! \echo{pop subap, pop pop subap}
    ^Gyere, gyere, gyere és rázd a bal ke^zed! \echo{pop subap, pop pop subap}
    ^Gyere, gyere, gyere és énekelj, hogy ^jól szóljon ^ a dalunk:
  \endverse

  \beginverse*
    Refrén
  \endverse

  \beginverse*
    \[C] Amikor ti hívtatok, \[G]úgy éreztem, kő vagyok,
    \[C]De itt rátok találtam, mit \[D]úgy, úgy, de \[D#]úgy vár\[D]tam!
  \endverse

  \beginverse
    ^Gyere, gyere, gyere és táncoljál velünk! \echo{pop subap, pop pop subap}
    ^Gyere, gyere, gyere és táncoljál velü^nk! \echo{pop subap, pop pop subap}
    ^Gyere, gyere, gyere és énekelj, hogy ^jól szóljon ^ a dalunk:
  \endverse

  \beginverse*
    Refrén
  \endverse

\endsong

\beginsong{Ároni áldás}[]

  \ifchorded
    \beginverse*
      {\nolyrics Előjáték: \[C] \[Em] \[Dm] \[G] \[C] \[G] \[C]}
    \endverse
  \fi

  \beginverse*\memorize
    \[C]Áldjon m\_{e}g \[Em]téged az {\[Dm G]{\_{Ú}}}r!
    \[C]Áldjon m\_{e}g \[Em]téged az {\[Dm G]{\_{Ú}}}r!
    És \[C]őrizzen \[G]meg tége\[C]det! \[C7]
  \endverse

  \beginverse*
    /: \[Dm]Világo\[Em]sítsa meg az \[C]{\_{Ú}}r
    Az \[Dm]Ő or\[Em]cáját te\[C]rajtad,
    És \[F]könyörüljön teraj\[G]tad! \[G7]
    \[C]Fordítsa az \[Em]Úr az Ő \[Dm]orcáját \[G]rád!
    \[C]Fordítsa az \[Em]Úr az Ő \[Dm]orcáját \[G]rád!
    És \[C]adjon bé\[G]két tené\[C]ked! :/ \rep{2} \echo{És \[C]adjon bé\[G]két tené\[C]ked!}
  \endverse

\endsong

\beginsong{Atya, Teremtő szeretet Lelke}
[
  index={Áradj szét!}
]

  \beginverse\memorize
    \[F]Atya, Teremtő \[A#]szeretet Lelke, \[Am] áradj \[C7]szét!
    \[F]Fiú, Megváltó \[A#]szeretet Lelke, \[Am] áradj \[C7]szét!
    \[F]Egyházvezető \[A#]szeretet Lelke, \[Am] áradj \[C7]szét!
    \[F]Szentség pecsétje, \[A#]szeretet Lelke, \[C7] áradj \[F]szét!
  \endverse

  \beginchorus
    \[F]Áradj \[Am]szét, Lélek, \[A#] áradj \[C7]szét!
    \[F]Áradj \[Am]szét, s töltsd be \[A#]minden \[C7]hívő szí\[F]vét!
  \endchorus

  \beginverse
    ^Szívből jövő ^imádság Lelke, ^ áradj ^szét!
    ^Mélyből fakadó ^sóhaj Lelke, ^ áradj ^szét!
    ^Isteni üzenet ^dicső Lelke, ^ áradj ^szét!
    ^Megbocsátás ^irgalmas Lelke, ^ áradj ^szét!
  \endverse

  \beginverse*
    Refrén
  \endverse

  \beginverse
    ^Szabadságom ^tüzes Lelke, ^ áradj ^szét!
    ^Imádságom ^kitartó Lelke, ^ áradj ^szét!
    ^Szenvedésem ^tűrő Lelke, ^ áradj ^szét!
    ^Békességem ^gyönyörű Lelke, ^ áradj ^szét!
  \endverse

  \beginverse*
    Refrén
  \endverse

  \beginverse
    ^Dicsőítlek, ^Atya Lelke, ^ szent kegye^lem!
    ^Magasztallak, ^Fiú Lelke, ^ örök örö^möm!
    ^Érints meg engem ^Isten Lelke, ^legyél éle^tem!
    ^Tarts meg engem, ^szeretet Lelke, ^Te vagy Iste^nem!
  \endverse

  \beginverse*
    Refrén
  \endverse

\endsong

\beginsong{Atyám két kezedben}[]

  \beginverse\memorize
    \[Dm]Atyám, két ke\[C]zedben, \[A7]csak ott lakha\[Dm]tom,
    \[F]Biztonságot \[C]csak Tőled ka\[Dm]pok. \[G]
    \[Dm]Újjá így te\[C]remtesz, \[A7]sebem ápo\[Dm]lod,
    \[F]Boldogság, hogy \[C]itt van ottho\[Dm]nom. \[G]
  \endverse

  \beginchorus
    /: \[F]Tarts meg két ke\[C]zedben, \[A7]őrizz meg U\[Dm]ram,
    \[F]Oltalmadban \[C]rejtsd el sorso\[Dm]mat! \[G] :/ \rep{2}
  \endchorus

  \beginverse
    ^Atyám, két ke^zedben, ^teljes az ö^röm,
    ^Ajándékod^ban gyönyörkö^döm! ^
    ^Tékozlóként ^éltem, ^tárva most ka^rod,
    ^Hűtlenségem ^nem hánytorga^tod! ^
  \endverse

  \beginverse*
    Refrén
  \endverse

  \beginverse
    ^Atyám, két ke^zedben, ^bátran sírha^tok,
    ^Fájdalmaim ^hordozod, ^tudom. ^
    ^Ott fenn a ke^reszten ^áldó két ke^zed,
    ^Bűneimmel ^én szegeztem ^fel! ^
  \endverse

  \beginverse*
    Refrén \ifchorded{\nolyrics (A legvégén \[G] nélkül.)}\fi
  \endverse

\endsong

\beginsong{Atyám, Tied vagyok}[]

  \beginverse\memorize
    \[C]Atyám, Tied \[F]vagyok, és \[G]szüntelen i\[C]mádlak!
    \[C]Dicsőség \[F]szent neved\[G4]nek! \[G7]
  \endverse

  \beginchorus
    \[C]Dicsőség ne\[F]vednek, \[E]dicsőség ne\[Am]vednek,
    \[F]Dicsőség \[G]szent neved\[F]nek! \[C]
  \endchorus

  \beginverse
    ^Jézus, Tied ^vagyok, és ^szüntelen i^mádlak!
    ^Dicsőség ^szent neved^nek! ^
  \endverse

  \beginverse*
    Refrén
  \endverse

  \beginverse
    ^Lélek, Tied ^vagyok, és ^szüntelen i^mádlak!
    ^Dicsőség ^szent neved^nek! ^
  \endverse

  \beginverse*
    Refrén \rep{2}
  \endverse

\endsong

\beginsong{A béke napja közel}
[
  by={Borka Zsolt}
]

  \beginverse*
    A \[F]béke \[C]napja köz{\[Dm A#]{\_{e}}}l, \[F]nézd, már \[A#]fénylik a \[C]jel! \[C7]
    \[F]Jézus \[A]szíve a \[A#]Földön, az \[F]é\[C]gen \[F]mindent \[C]egybeöl\[F]el.
  \endverse

\endsong

\beginsong{Bízom Benned - !!}[]

  \ifchorded
    \beginverse*
      {\nolyrics Előjáték: \[G] \[C9] \[Em7] \[D]}
    \endverse
  \fi

  \beginchorus
    \[G]Bízom \[C9]Benned, \[Em7]igazságos \[D]Úr vagy,
    \[G]Bízom \[C9]Benned, \[Em7]irgalmas, \[D]j{\_{ó}}!
    \[G]Bízom \[C9]Benned \[Em7]mind{\_{e}}n\[D]nap, örö\[G]kké! \[C9] \[Em7] \[D]
  \endchorus

  \beginverse\memorize
    \[G] Olyan messziről jöttem \[Am7]én,
    És Te \[F]megvár\[C]tál, sokat \[G] vártál \[D]r{\_{á}}m,
    \[G] Úgy hívlak a magány éjje\[Am7]lén,
    Mikor \[F]elvész a \[C]szó, mikor csak \[G]átölelni \[D]jó!
    \[Am]Tudom, hogy \[Em]távol voltam, \[G]fájt Ne\[D]ked.
    \[Am]Tudom, hogy \[Em]nélküled \[C]nem élhe\[D]t{\_{e}}k.
  \endverse

  \beginverse*
    Refrén
  \endverse

  \beginverse
    ^ Nézd, a szívem már Ti^éd!
    Olyan ^boldog^ság, mikor ^ Téged ^l{\_{á}}t!
    ^ Hallod hangom lágy ne^szét,
    Kérlek, ^szólj hoz^zám, mikor csak ^Téged hív a ^szám!
    ^Életem ^nem féltem, hisz ^jó Ve^led.
    ^Tudom, hogy ^nélküled ^nem élhe^t{\_{e}}k!
  \endverse

  \beginverse*
    Refrén \rep{2} \[G]
  \endverse

\endsong

\beginsong{Boldogság, öröm - !!!!!!}[]

  \ifchorded
    \beginverse*
      {\nolyrics Előjáték: /: \[G] \[Am] \[D-D7] \[G!] \[D7!] :/ \rep{2}}
    \endverse
  \fi

  \beginverse\memorize
    Ha \[G]valaki látja az \[Am]arcomat,
    Ha \[D]valaki hallja a \[G]hangomat,
    \[E]Megtudja-e rólam, hogy \[Am]Jézus él,
    És \[D]boldog-e az \[D7]ember, ha \[G]szeretetben él?
  \endverse

  \beginchorus
    \[G]Boldogság, öröm az élete\[Am]m,
    \[D]Akárhová \[D7]megyek, kell, hogy vi\[G]gyem.
    \[E]Örömöm Jézus adta ne\[Am]kem,
    Hogy \[D]ne csak az e\[D7]nyém, de \[D#7]mindenkié \[D7]legye\[G]n! \[F#] \[F]
  \endchorus

  \beginverse
    Ha ^valaki nézi a ^két szemem,
    Ha ^valaki fogja a ^két kezem,
    ^Megtudja-e rólam, hogy ^Jézus vár,
    És ^boldog-e az ^ember, ha ^az Ő útján jár?
  \endverse

  \beginverse*
    Refrén
  \endverse

\endsong

\beginsong{Csak Jézusnak szolgálok}[]

  \beginchorus
    Csak \[G]Jézus\[D]nak \[Em]szolg\_{á}\[C]lok, az \[G]Ő or\[C]szágát \[G]v{\_{á}}\[D]rom,
    Mert \[G]eljön \[D]Ő, az \[C]Üdvözí\[D#]tő, ó \[G]Jézus, \[D]feltámadt Ki\[G]rály! \[C] \[G]
  \endchorus

  \beginverse\memorize
    \[Em]Áldjad, lelkem \[Hm]Őt, \[Em]dicsérd Terem\[Hm]tőd, \[G]most és \[C]mindörökké, \[G]{\_{á}}\[D]men!
    \[Em]Áldjad, minden \[Hm]nép, \[C]dicsérd szent Ne\[G]vét, \[G]most és \[C]mindörökké, \[D]{\_{á}}\[G]men!
  \endverse

  \beginverse
    ^Hálát adja^tok, ^áldást mondja^tok, ^most és ^mindörökké, ^{\_{á}}^men!
    ^Egyházad di^csér ^nagy jóságo^dért, ^most és ^mindörökké, ^{\_{á}}^men!
  \endverse

  \beginverse
    ^Erős vár az ^Úr, ^menedéket ^nyújt, ^most és ^mindörökké, ^{\_{á}}^men!
    ^Áldott irga^lom, ^véred olta^lom, ^most és ^mindörökké, ^{\_{á}}^men!
  \endverse

  \beginverse
    ^Hatalmas az ^Úr, ^jóságos az ^Úr, ^most és ^mindörökké, ^{\_{á}}^men!
    ^Szívem áldja ^Őt, ^ujjongjon a ^Föld, ^most és ^mindörökké, ^{\_{á}}^men!
  \endverse

  \beginverse
    ^Szentek, angya^lok, ^szeráfkóru^sok, ^most és ^mindörökké, ^{\_{á}}^men!
    ^Néked zenge^nek, ^így ünnepel^nek, ^most és ^mindörökké, ^{\_{á}}^men!
  \endverse

  \beginverse
    ^Áldott légy ^Atya, ^s egyszülött ^Fia, ^most és ^mindörökké, ^{\_{á}}^men!
    ^Éltető Lé^lek, ^Téged dicsér^lek, ^most és ^mindörökké, ^{\_{á}}^men!
  \endverse

\endsong

\beginsong{Ébredj ember}
[
  li={https://youtu.be/-DGzHCfmv5k alapján.}
]

  \ifchorded
    \versesep=12pt plus 3pt minus 7pt
  \fi

  \ifchorded
    \beginverse*
      {\nolyrics Előjáték: \[C] \[F] \[C] \[C] \[F] \[G] \[Am] \[F] \[C] \[Em] \[Am] \[Dm] \[G] \[C]}
    \endverse
  \fi

  \beginverse
    Ébredj, \[C]ember, \[G] mély álmod\[Am]ból, \[F] Jézus \[C]megment \[Dm] rabságod\[G]ból!
    Hogyha \[Am]hívod, \[Em] eljön \[F]Ő, egész \[C]bensőm, \[Dm] imádjad \[G]Őt!
  \endverse

  \beginchorus
    Alle\[C]luja, alle\[F]l{\_{u}}\[C]ja!
    Alle\[C]luja, alle\[F]l{\_{u}}\[G]ja!
    Alle\[Am]luja, alle\[F]luja, al\[C]le\[Em]l{\_{u}}\[Am]ja!
    Alle\[Dm]luja, alle\[G]l{\_{u}}\[C]ja!
  \endchorus

  \beginverse*
    {\nolyrics (Két félhang emelkedés.) \[A7]}
  \endverse

%   \textnote{Két félhang emelkedés.}

  \beginchorus
    Alle\[D]luja, alle\[G]l{\_{u}}\[D]ja!
    Alle\[F#m]luja, alle\[G]l{\_{u}}\[A]ja!
    Alle\[Hm]luja, alle\[G]luja, al\[D]le\[F#]l{\_{u}}\[Hm]ja!
    Alle\[Em]luja, alle\[A]l{\_{u}}\[D]ja!
  \endchorus

  \beginverse\memorize
    Áldjad, \[D]lelkem, \[A] az Úr ne\[Hm]vét, \[G] mert meg\[D]tartja \[Em] ígére\[A]tét!
    Mennyi \[Hm]jót tett \[F#m] az Úr ve\[G]lem, Reá \[D]bízom \[Em] az éle\[A]tem.
  \endverse

%   \beginverse
%     Föltámadt Jézus, győztes király, elhozta nékünk az országát.
%     Jöjj Szentlélek, szállj közénk, hozz a mennyből szép, tiszta fényt!
%   \endverse

  \beginverse*
    Refrén
  \endverse

  \beginverse
    És az ^élet, ^íme, megjele^nt, ^ köztünk ^él Ő ^ Szentlelké^ben.
    Hív Ő ^téged, ^ hogy élj ve^le, add át ^néki ^ a szíve^det!
  \endverse

  \beginverse*
    Refrén
  \endverse

\endsong

\beginsong{Együtt fog ujjongni}
[
  sr={Jeremiás 31,13}
]

  \beginverse*
    \[H7]Együtt fog ujjongni \[Em]fiatal és öreg is
    És \[Am]örömében táncot lejt a \[H7]l{\_{á}}ny.
  \endverse

  \beginverse*
    % {\nolyrics  \[H7] \[Em] \[Am] \[H7]}

    \[H7]La, la la la, la la \[Em]la, la-la-la la la la
    La \[Am]la la la la, la la la la, \[H7]la la la la
  \endverse

  \beginverse*
    \[H7]Gyászukat ünnepre fordítom, s nem búsulnak ezután!
    Vi\[Am]dámmá te\[D]szem a szívü\[G]ket,
    Hogy \[Em]örvendezze\[Am]nek a \[H7]bánatuk he\[Em]ly{\_{e}}tt!
    Vi\[Am]dámmá te\[D]szem a szívü\[G]ket,
    Hogy \[Em]örvendezze\[Am]nek a \[H7]bánatuk he\[Em]lyett!
  \endverse

\endsong

\beginsong{Én Uram, én Istenem}[]

  \ifchorded
    \beginverse*
      {\nolyrics Előjáték: \[Em] \[H7] \[G] \[D] \[Em] \[H7] \[G] \[D] \[Em]}
    \endverse
  \fi

  \beginverse\memorize
    \[Em]Én Uram, én \[H7]Istenem,
    \[G]Vedd el tőlem \[D]mindenem,
    Ami \[Am]gátol \[Am7] Fe\[Em]léd!
  \endverse

  \beginverse
    ^Én Uram, én ^Istenem
    ^Add meg nekem ^mindenem,
    Ami ^segít ^ Fe^léd!
  \endverse

  \beginverse
    \[Em]Én Uram, én \[H7]Istenem,
    \[G]Fogadd el az \[D]életem,
    \[C]Hadd legyen e\[D]gészen a Ti{\[G H7]é}d!
  \endverse

  \beginverse
    \[Em]Én Uram, én \[H7]Istenem,
    \[G]Fogadd el az \[D]életem,
    \[C]Hadd legyen e\[D]gészen a Ti\[Em]éd! \[H7]
  \endverse

  \ifchorded
    \beginverse*
      {\nolyrics \[G] \[D] \[Am] \[Am7] \[Em]}
    \endverse
  \fi

\endsong


\beginsong{A fény, ami bennem ég - !!}[]

  \beginverse\memorize
    A \[G]fény, ami bennem ég nem fog már kihunyni,
    A \[C]fényt, ami bennem ég, \[G]égni hagyom már. \echo{Hej babám!}
    A fény, ami bennem ég \[H7]sugározzon \[Em]szét,
    Legyen \[G]fény, legyen \[D]fény, legyen \[G]fény! \[C] \[G] \[D]
  \endverse

  \beginverse
    A fény, ami Krisztusé…
  \endverse

  \beginverse
    A ^világ minden sarkában fénynek kell kigyúlni,
    A ^világ minden sarkában ^éghetne a láng. \echo{Hej babám!}
    A világ minden sarkába a ^fényt elvihet^ném,
    Legyen ^fény, legyen ^fény, legyen ^fény! ^ ^ ^
  \endverse

  \beginverse
    A Szent Imre minden sarkában…
  \endverse

  \beginverse
    Budaörs minden sarkában…
  \endverse

  \beginverse
    A Szent Ferenc minden sarkában…
  \endverse

  \beginverse
    Szentendre minden sarkában…
  \endverse

  \beginverse
    Fehérvár minden sarában…
  \endverse

  \beginverse
    Kanizsa minden sarkában…
  \endverse

  \beginverse
    Szeged minden sarkában…
  \endverse

  \beginverse
    Kaposvár minden sarkában…
  \endverse

  \beginverse
    Pécs minden sarkában…
  \endverse

\endsong

\beginsong{Föltámadt! Alleluja!}
[
  by={Sillye Jenő},
  index={Álltok szótlanul}
]

  % \ifchorded
  %   \beginverse*
  %     {\nolyrics Előjáték: \[E] \[A] \[A7]}
  %   \endverse
  % \fi

  \beginverse\memorize
    \[D]Álltok \[A7]szótla\[D]nul, a \[G]dárda \[A]földre \[D]hull, \[Hm]
    Itt \[G]más e\[A]rő az {\[D]\[Hm]Ú}r, itt \[Em]más e\[A]rő az \[D]Úr!
  \endverse

  \beginverse
    Az ^éj az ^hosszú ^volt, de ^eltűnt ^már a ^Hold, ^
    A ^hajnal {^felsik}{^^o}lt, a ^hajnal ^felsi^kolt: \[D7]
  \endverse

  \beginchorus
    \[G]Föltá\[A]madt! Allel{\[D]\[Hm]u}ja! \[Em]Föltámadt! \[A7]Allelu\[D]ja! \[Hm]
    \[G]Föltá\[A]madt! Allel{\[D]\[Hm]u}ja! \[Em]Föltámadt! \[A7]Allelu\[D]ja!
  \endchorus

  \beginverse
    ^Elég volt ^már, e^lég, ^fénnyel ^hinti az ^ég ^
    Az ^új nap {^reggel}{^^é}t, az ^új nap ^regge^lét.
  \endverse

  \beginverse
    ^Virágban ^áll a ^rét, a ^virágos ^úton ^át ^
    ^Valaki ^jön fel{^^é}d, ^valaki ^jön fe^léd. \[D7]
  \endverse

  \beginverse*
    Refrén
  \endverse

  \beginverse
    ^Virágban ^áll a ^rét, a ^virágos ^úton ^át ^
    ^Jézus ^jön fel{^^é}d, ^Jézus ^jön fe^léd.
  \endverse

  \beginverse*
    Refrén \rep{2}
  \endverse

\endsong

\beginsong{Ha jön az Úr}[]

  \ifchorded
    \beginverse*
      {\nolyrics Előjáték: \[E] \[E7] \[A] \[Am] \[E] \[H7] \[E] \[A] \[E]}
    \endverse
  \fi

  \beginverse\memorize
    Ha jön az \[E]Úr, ha visszatér, ha jön az Úr, ha vissza\[H7]tér,
    Hívj, A\[E]tyám, a \[E7]szentek \[A]közé, \[Am] ha jön az \[E]Úr, ha \[H7]vissza\[E]tér! \[A] \[E!]
  \endverse

  \beginverse
    Ha minden ^szent életre kél, ha minden szent életre ^kél,
    Hívj, A^tyám, a ^szentek ^közé, ^ ha minden ^szent é^letre ^kél! ^ ^
  \endverse

  \beginverse
    Ha zengik ^mind, alleluja, ha zengik mind, allelu^ja,
    Hívj, A^tyám, a ^szentek ^közé, ^ ha zengik ^mind al^lelu^ja! ^ ^
  \endverse

  \beginverse
    Ha ég és ^Föld új arcot ölt, ha ég és Föld új arcot ^ölt,
    Hívj, A^tyám, a ^szentek ^közé, ^ ha ég és ^Föld új ^arcot ^ölt! ^ ^
  \endverse

  \beginverse
    A mennyben ^fenn, oly' nagy a csend, elvétve egy-két hárfa ^zeng,
    De nem így ^lesz, ó, ^alle^luja, ^ ha egyszer ^én is ^ott le^szek! ^ ^
  \endverse

  \beginverse
    Nem sakko^zik, nem golfozik, egy rendes angyal nem i^szik,
    Így éne^kel, ó ^alle^luja, ^ miközben ^rock n' ^rollo^zik! ^ ^
  \endverse

\endsong

\beginsong{Hálát adok az esti órán}
[
  ititle={Esti hála}
]

  % / jelek után: https://www.youtube.com/watch?v=ggrTwNKP5Ew

  \beginverse\memorize
    \[D]Hálá\[Hm]t adok az \[Em]esti \[A6]órán,
    \[D]Hálá\[D7]t adok, hogy \[G]itt az \[A7]éj,
    \[D]Hálá\[D7]t adok, hogy \[G]szívem \[Gm]mélyén
    \[D]Hála \[A7]dala \[D]kél.
  \endverse

  \beginverse
    ^Hálá^t adok a ^csillag^fényért,
    ^Hálá^t a sűrű ^éjje^lért,
    ^Hálá^t adok, hogy ^nem vagy ^messze,
    ^Most is ^küldesz ^fényt.
  \endverse

  \beginverse
    ^Hálá^t adok, hogy ^küldtél ^testvért
    ^Társkén^t, aki ma ^mellém ^állt.
    ^Hálá^t adok, hogy ^megen^gedted
    ^Azt is, ^ami ^fájt.
  \endverse

  \beginverse
    ^Hálá^t adok, hogy ^Tested ^táplált,
    % Igéd
    ^Hálá^t, hogy ma is ^szólt sza^vad.
    ^Hálá^t adok, hogy ^lelkem ^biztos
    ^Cél fe^lé ha^lad.
  \endverse

  \beginverse
    ^Hálá^t adok, hogy ^mellém ^álltál,
    ^Hálá^t, hogy szép az ^ottho^nom.
    % van jó?
    ^Hálá^t, hogy munkám ^elvé^geztem,
    ^S békén ^alha^tom.
  \endverse

  \beginverse
    ^Hálá^t adok a ^sok-sok ^jóért,
    ^Hálá^t adok, hogy ^megbo^csátsz.
    ^Hálá^t adok, hogy ^el nem ^hagytál,
    ^S újabb ^útra ^vársz.
  \endverse

  \beginverse
    ^Hálá^t adok, hogy ^szemed ^rajtam,
    ^Hálá^t adok, hogy ^reményt ^adsz.
    ^Hálá^t adok, hogy ^biztos ^eljössz,
    ^Mint a ^virra^dat.
  \endverse

\endsong

\beginsong{Hálát adok, hogy itt a reggel}[]

  \beginverse\memorize
    \[D]Hálá\[Hm]t adok, hogy \[Em]itt a \[A6]reggel,
    \[D]Hálá\[D7]t adok az \[G]új na\[A7]pon.
    \[D]Hálá\[D7]t adok, hogy \[G]minden \[Gm]percem
    \[D]Néked \[A7]adha\[D]tom.
  \endverse

  \beginverse
    ^Hálá^t nem csak a ^jó test^vérért,
    ^Hálá^t mindenki^ért a^dok.
    ^Hálá^t adok, hogy ^minden ^sértést
    ^Megbo^csátha^tok.
  \endverse

  \beginverse
    ^Hálá^t adok, hogy ^munkát ^küldesz,
    ^Hálá^t adok, hogy ^fény ra^gyog.
    ^Hálá^t adok, hogy ^lelkem ^fényes
    ^És bol^dog va^gyok.
  \endverse

  \beginverse
    ^Hálá^t adok a ^vidám ^percért,
    ^Hálá^t, ha szomo^rú le^szek.
    ^Hálá^t adok, hogy ^adsz e^lém célt
    ^És ke^zed ve^zet.
  \endverse

  \beginverse
    ^Hálá^t adok, hogy ^hallom ^hangod,
    ^Hálá^t adok a ^jó hí^rért.
    ^Hálá^t adok, hogy ^Tested ^adtad
    ^Minden ^embe^rért.
  \endverse

  \beginverse
    ^Hálá^t adok az ^üdvös^ségért,
    ^Hálá^t adok, hogy ^várni ^fogsz.
    ^Hálá^t adok, hogy ^ma reg^gel még
    ^Hálát ^adha^tok.
  \endverse

\endsong

\beginsong{Hús-vér templom}
[
by={Pintér Béla}
]

  \ifchorded
    \beginverse*
      {\nolyrics Előjáték: /: \[Dm] \[A#] \[C] \[F] :/ \rep{2}}
      % TODO: A# helyett Gm?
    \endverse
  \fi

  \beginchorus
    /: Nekem \[Dm]nincs más \[Gm]rajtad \[C]kívül \[F]Jézus,
    \[Dm]Éle\[Gm]tem for\[C]rása \[F]vagy.
    \[Dm]Kézzel, \[Gm7]lábbal, \[A]szívvel, \[Dm]szájjal
    \[A#]Dicsérlek Uram. \[A4] \[A] :/ \rep{2}
  \endchorus

  \beginverse
    Ez a \[Dm]hús-vér \[Gm]templom \[A]Érted \[Dm]épült,
    Neked \[A#]ég a \[Gm7]tűz, bent \[A]oltárai\[Dm]nál.
    Ez a hús-vér \[Gm]templom \[A]Téged \[Dm]dicsér,
    Szívem \[A#]minden \[Gm7]húrja \[A]Rólad muzsi\[Dm]kál.
  \endverse

  \beginverse*
    Refrén
  \endverse

\endsong

\beginsong{Ím, itt jön Ő}[]

  \versesep=12pt plus 3pt minus 7pt

  \ifchorded
    \beginverse*
      {\nolyrics Előjáték: \[Em] \[Hm] \[Em] \[C] \[Hm] \[Em]}
    \endverse
  \fi

  \beginchorus
    /: Ím, \[Em]itt jön Ő, átkel \[Hm]a hegyen,
    a \[Em]dombon át sza\[C]lad felém a \[Hm]kedve\[Em]sem! :/ \rep{2}
  \endchorus

  \beginverse\memorize
    Hát \[G]jöjj, Uram, lásd, Ti\[D]éd a szívem,
    A \[Em]kerted Téged vár, a szőlőd \[H]is terem.
    A \[G]szív dalol, zengi \[D]énekét:
    A \[Em]lelkem ég, ó \[C]Istenem, sze\[Hm]relme\[Em]dért.
  \endverse

  \beginverse*
    Refrén
  \endverse

  \beginverse
    Úgy ^vágyom, Uram, ünne^pelni Veled,
    Meg^részegít a bor s a méz, mit ^adsz nekem.
    Ha ^elfogadod minden ^kincsemet,
    Én ^átadom az ^életem, Ti^éd le^gyen!
  \endverse

  \beginverse*
    Refrén
  \endverse

  \beginverse
    Az ^Úr elé együtt ^indulunk,
    Ő ^mindig újra vár miránk, ha ^elbukunk.
    A ^nász idején ünne^peljetek,
    Hisz ^itt a vőle^gény, ki minket ^úgy sze^ret!
  \endverse

  \beginverse*
    Refrén \echo{A \[Hm]kedve\[Em]sem, a \[Hm]ked\[Hm7]ve\[Em]sem.}
  \endverse

\endsong

\beginsong{Indulj a világba}[]

  \ifchorded
    \beginverse*
      {\nolyrics Előjáték: /: \[D] \[D] \[D] \[D-D&] :/ \rep{2}}
    \endverse
  \fi

  \beginverse\memorize
    \[D]Én vagyok, ki a közösséget építi,
    És \[A]én vagyok, ki a közösséget építi,
    És \[D]én vagyok, ki a közösséget építi,
    És \[A]tovább szeretném ad\[D]ni!
  \endverse

  \beginchorus
    \[G]Indulj a világba, s \[D]menjél bármerre:
    \[A]magas havasokba, s a kék \[D]tengerre,
    \[G]Indulj a világba, az \[D]Úr van teveled,
    \[A]Indulj hát és építsd a kö\[D]zös\[A]sé\[D]get!
  \endchorus

  \beginverse
    ^Te vagy az, ki a közösséget építi,
    És ^te vagy az, ki a közösséget építi,
    És ^te vagy az, ki a közösséget építi,
    És ^tovább szeretné ad^ni!
  \endverse

  \beginverse
    ^Mi vagyunk, kik a közösséget építjük,
    És ^mi vagyunk, kik a közösséget építjük,
    És ^mi vagyunk, kik a közösséget építjük,
    És ^tovább szeretnénk ad^ni!
  \endverse

  \beginverse
    ^Az Úr, az Úr, ki a közösséget építi,
    ^Az Úr, az Úr, ki a közösséget építi,
    ^Az Úr, az Úr, ki a közösséget építi,
    És ^tovább szeretné ad^ni!
  \endverse

  \beginverse
    ^A Szeretet, a Szeretet, ki a közösséget építi,
    ^A Szeretet, a Szeretet, ki a közösséget építi,
    ^A Szeretet, a Szeretet, ki a közösséget építi,
    És ^tovább szeretné ad^ni!
  \endverse

  \beginverse
    ^Az Antioch, az Antioch, ki a közösséget építi,
    ^Az Antioch, az Antioch, ki a közösséget építi,
    ^Az Antioch, az Antioch, ki a közösséget építi,
    És ^tovább szeretné ad^ni!
  \endverse

  \beginverse
    ^És én, és te, és mi, az Úr, a Szeretet, az Antioch, ki a közösséget építi,
    És ^én, és te, és mi, az Úr, a Szeretet, az Antioch, ki a közösséget építi,
    És ^én, és te, és mi, az Úr, a Szeretet, az Antioch, ki a közösséget építi,
    És ^tovább szeretné ad^ni!
  \endverse

\endsong

\beginsong{Indulj és menj}
[
  sr={Ezekiel 3,4},
  by={Hollósy Péter}
]

  \ifchorded
    \beginverse*
      {\nolyrics Előjáték: /: \[C] \[D] \[Em] \[H7]} :/ \rep{2}
    \endverse
  \fi

  \beginverse\memorize
    \[Em]Indulj és \[G]menj, \[D]hirdesd sza\[Em]vam,
    \[C]Népemhez \[D]küldelek \[Em]én! \[H7]
    \[Em]Tövis és \[G]gaz, \[D]vér és pa\[Em]nasz,
    \[C]Meddig hall\[D]gassam még \[Em]én?
  \endverse

  \beginchorus
    \[G]Küldelek \[D]én, \[H7]megáldlak \[Em]én,
    Csak \[C]menj és \[D]hirdesd sza\[Em]vam! \[H7]
    \[G]Küldelek \[D]én, \[H7]megáldlak \[Em]én,
    Csak \[C]menj és \[D]hirdesd sza\[Em]vam!
  \endchorus

  \beginverse
    ^Tüzessé ^teszem ^ajkai^dat,
    ^Gyémánttá ^homloko^dat. ^
    ^Népemnek ^őrévé ^rendellek ^én,
    ^Lelkemet ^adom mel^léd!
  \endverse

  \beginverse*
    Refrén
  \endverse

\endsong

\beginsong{Itt vagyok most, jó Uram - !!}[]

  \beginverse\memorize
    \[D]Itt vagyok most, jó Uram, \[G]Hozzád száll e dal.
    \[D]Tudom, sokszor hívtalak, de ma \[G]másképp szól e hang
    mert ma \[Em]nem azt kérem, hogy \[A]adj még, nem kö\[D]nyörgök: figyelj r\[G]ám,
    csak \[D]azt mon\[Hm7]dom, hogy \[Em7]szeret\[A7]lek, Aty\[D]ám! \[D7]
  \endverse

  \beginchorus
    Szeret\[Em7]lek, ó, U\[A7]ram, hálám \[D]szívemből fak\[G]ad, hiszen
    \[Em7]mindent látsz Te ben\[A]nem, \[D]hatalmad olyan \[D7]nagy
    Szeret\[Em7]lek, ó, U\[A]ram, hálám \[D]szívemből fak\[G]ad
    és \[D]ének\[Hm7]lem, hogy \[Em7]áldott \[A7]légy U\[D]ram!
  \endchorus

  \beginverse
    ^Ismét térden állok én, s ^tudom hogy hallgatsz rám.
    Bár ^sokszor hideg szavakat szól ^kopott és halk imám,
    De most ^el kell, hogy mondjam egy^szer, amit ^oly rég szeretn^ék,
    hogy ^szeretl^ek, a ^szívem csak ti^éd! ^ ^
  \endverse

  \beginverse*
    Refrén
  \endverse

\endsong

\beginsong{Jézus a mi oltalmunk}[]

  \versesep=12pt plus 3pt minus 8pt

  \beginchorus
    \[Dm]Jézus a mi \[A]oltalmunk, \[Dm]erőssé\[C]günk,
    Ha \[F]ránk szakad \[C]minden baj, \[Dm]mégse \[A]fé\[Dm]lünk!
    \[Dm]Ő a mi \[A]páncélunk, \[Dm]erős paj\[C]zsunk,
    Ha \[F]ránk tör az \[C]ellenség, \[Dm]Benne \[A]bí\[Dm]zunk!
  \endchorus

  \beginverse\memorize
    \[F]Ez a föld szét\[C]hullik, \[Dm]minden hegy el\[A]omlik,
    \[B]De egy \[F]kőszikla \[C]örökké \[F]áll!
    \[F]Tajtékzik a \[C]tenger, \[Dm]gyilkol a sok \[A]fegyver,
    \[B]Né\[C]künk \[F]mégsem árt \[A]a ha\[Dm]lál!
  \endverse

  \beginverse*
    Refrén
  \endverse

  \beginverse
    ^Eladó már ^minden, ^örömöd még ^sincsen,
    ^Ha ez a ^forrás ^nem árad ^rád!
    ^Ő az, aki ^ingyen ^od'adott már ^mindent,
    ^Ér^tünk ^áldozva ^Önma^gát!
  \endverse

  \beginverse*
    Refrén
  \endverse

  \beginverse
    ^Fogták, elí^télték ^és megfeszí^tették,
    ^Isten ^Fiát mi^helyet^tünk!
    ^Nincs már í^télet, ^vád sohasem ^érhet,
    ^Vé^re ^eltörölt ^minden ^bűnt!
  \endverse

  \beginverse*
    Refrén
  \endverse

\endsong

\beginsong{Jézus életem, erőm, békém}
[
  by={Taizé / Jacques Berthier}
]
  \beginverse*
    \[Dm]Jé\[C]zus \[F]életem, erőm, \[A#]bé\[C]kém,
    \[Dm]Jé\[C]zus \[F]társam, \[Dm]örö\[C]möm,
    Benned \[A#]bízom, \[A]Te vagy az \[Dm]Úr;
    Már \[C]nincs mit \[F]félnem, mert \[A#]bennem \[C]élsz,
    Már \[Am]nincs mit \[Dm]félnem, mert \[A#]ben\[C]nem \[F]élsz.
  \endverse

\endsong

\beginsong{Jézus néz rám}[]

  \ifchorded
    \beginverse*
      {
        \nolyrics Előjáték:
        \[C] \[Dm] \[Am] \[E7] \[Dm] \[G7] \[C] \[G7]
        \[C] \[Dm] \[Am] \[E7] \[Dm] \[G7] \[C]
      }
    \endverse
  \fi

  \beginverse
    \[C]Jézus \[Dm]néz rám a \[Am]két szemed\[E7]ből, arra \[Dm]kér, arra \[G7]vár, hogy megért\[C]sem, \[G7]
    \[C]Jézus \[Dm]néz rám a \[Am]két szemed\[E7]ből, arra \[Dm]kér, arra \[G7]vár, hogy szeres\[C]sem.
  \endverse

  \beginchorus
    /: \[C]Jézus \[H7]arra \[E]kér, mert \[C]bennünk \[H7]emberekben \[E]él, hogy
    \[C]meglás\[C7]sam és \[F]szeressem, ha \[C]eljön hozzám \[G7]valaki\[C]ben. \[G$^{\;(csak\;ismétlésnél)}$]{\ifchorded\hspace{0.4cm}\fi:/ \rep{2}}
    % vagy inkább \quad ?
  \endchorus

\endsong

\beginsong{Jézus vár terád}[]

  \beginchorus
    /: \[Em] Jézus vár ter\[Hm]ád, Lelkét küldi hozz\[C]ád,
    Mert közel jön \[D]Ő, irgalmas Üdvözít\[Em]ő. :/ \rep{2}
  \endchorus

  \beginverse\memorize
    \[Am]Jöjjetek Hoz\[D]zá, kik \[Hm]szomjazzátok \[C]Őt,
    \[Am]Bensejéből \[C]élő víz forrása fak\[H7]ad!
  \endverse

  \beginverse*
    Refrén
  \endverse

  \beginverse
    ^Jöjjetek Hoz^zá, kik ^megfáradtat^ok,
    ^Megkönnyíti ^Jézus nehéz terhet^ek!
  \endverse

  \beginverse*
    Refrén
  \endverse

  \beginverse
    ^Jöjjetek Hoz^zá, ti ^alázatos^ak,
    ^Nektek akarja ^adni dicső ország^át!
  \endverse

  \beginverse*
    Refrén \rep{2}
  \endverse

\endsong

\beginsong{Jó az Úrban bizakodni}[]

  \beginverse*
    \[Dm]Jó az Úrban \[A]bizakodni, \[Dm]j\[C]ó az \[F]Úr.
    \[Gm]Re\[C]mélj, és \[F]bízz \[C]ben\[Dm]ne, \[A#]j\[C]ó az \[Dm]Úr!
  \endverse

\endsong

\beginsong{Jöjj, hívunk, jöjj}[]

  \ifchorded
    \beginverse*
       {\nolyrics Előjáték: \[E] \[E] \[E] \[E] \[C#m] \[F#m] \[A] \[H] \[A] \[H]}
    \endverse
  \fi

  \beginverse\memorize
    \[E]Jöjj, hívunk, \[C#m]jöjj! Szentlé\[F#m]lek, a szí\[A]vünk\[H]be \[A]á\[H]radj!
    \[E]Jöjj, hívunk, \[C#m]jöjj! Szentlé\[F#m]lek, a szí\[A]vünk\[H]be \[A]á\[H]radj!
  \endverse

  \beginchorus
    Úgy \[A]vá\[E]runk és ki\[A]ál\[E]tunk:
    /: Áradj \[H]szét! \echo{Vigasztaló}
    Áradj \[A]szét! \echo{Szabadító}
    Áradj \[E]szét! :/ \rep{2}
  \endchorus

  \beginverse
    ^Jöjj, hívunk, ^jöjj! Mint fo^lyó, amely ^túl^á^ra^dó!
    ^Jöjj, hívunk, ^jöjj! Mint fo^lyó, amely ^túl^á^ra^dó!
  \endverse

  \beginverse*
    Refrén
  \endverse

\endsong

\beginsong{Jöjj, itt az idő}[]

  \ifchorded
    \beginverse*
       {\nolyrics Előjáték: \[D] \[D4] \[D] \[D4]}
    \endverse
  \fi

  \beginverse\memorize
    \[D]Jöjj, itt az idő, hogy \[D4]éb\[D]redj!
    \[A]Jöjj, itt az idő, hogy \[Em]áld\[F#m]juk \[G]Őt!
    \[D]Jöjj, szívedet add az \[D4]Úr\[D]nak!
    \[A]Jöjj, imádd az Istent \[Em]úgy, \[F#m]ahogy \[G]vagy!
  \endverse

  \beginverse*
    Hát \[D]jö\[D2 D4 D]jj!
  \endverse

  \beginchorus
    /: \[G]Minden nyelv megvallja, hogy \[D]Ő az Úr! \[G]És mindenki térdet \[D]hajt!
    De a \[G]legnagyobb kincs mégis \[Hm]azokra vár, kik \[Em]most Őt válasz\[A4]tják! \[A] :/ \rep{2}
  \endchorus

  \beginverse*
    1. versszak
  \endverse

  \beginverse*
    Hát \[D]jöjj! \[G] Ó \[D]jöjj! \[G] Hát \[D]jöjj!
  \endverse

\endsong

\beginsong{Kész az én szívem}
[
  sr={57. zsoltár alapján},
  by={Lázár Attila}
]

  \beginverse\memorize
    Kész az \[Dm]én sz\_{í}vem, Uram, \[Gm]Néked zeng a \[Dm]dalom.
    Elő hát hárfa, lant, hadd kel\[C]tsem fel a h\_{a}j\[A]nalt! \[A4] \[A]
  \endverse

  \beginchorus
    /: Szent az \[Dm]Úr, jó az Úr, hirde\[Gm]tem a népek \[Dm]között,
    Mert az Ő kegyel\[Gm]me felhőkig \[Dm]ér. \[A-A4$^{\;(csak\;ismétlésnél)}$]{\ifchorded\hspace{1cm}\fi:/ \rep{2}}
  \endchorus

  \beginverse
    Igaz^ság, hatalom igéd ^minden egyes ^szava.
    Szívemben rejtem el, soha ^el ne fel\_{e}j^tsem! ^ ^
  \endverse

  \beginchorus
    /: Szent az \[Dm]Úr, jó az Úr, hirde\[Gm]tem a népek \[Dm]között,
    Mert az Ő hűsé\[Gm]ge az égig \[Dm]ér. \[A-A4$^{\;(csak\;ismétlésnél)}$]{\ifchorded\hspace{1cm}\fi:/ \rep{2}}
  \endchorus

\endsong

\beginsong{Két kezemben}
[
  by={Pipó József},
  li={Kék könyv / 119.}
]

  \ifchorded
    \beginverse*
      {\nolyrics
        Előjáték:
        \[D] \[G] \[A] \[D]
        \[G] \[A] \[D]
      }
    \endverse
  \fi

  \beginverse*
    \[D]Két kezemben \[G]rejtem, őri\[D]zem az éle\[Em]tem, \[G] minde\[D]nem. \[A]
    \[D]Két kezemben \[G]titkon van je\[D]len a végte\[Em]len \[G] kegye\[D]lem.
    \[G]Ő az erőm, \[A]reményem, hi\[D]tem, \[Hm]Tőle kapom minden szerete\[Em]tem. \[A]
    \[D]Föl nem fogom, de \[G]hiszem szünte\[D]len: itt vagy ve\[Em]lem, \[A] Iste\[G]nem. \[G] \[A]
  \endverse

  \ifchorded
    \beginverse*
      {\nolyrics Utójáték:
        \[D] \[G] \[A] \[D]
        \[G] \[A] \[D]
      }
    \endverse
  \fi

\endsong

\beginsong{Kicsiny kis fényemmel}[]

  \beginverse\memorize
    \[G]Kicsiny kis fényemmel világítani fogok.
    \[C]Kicsiny kis fényemmel világítani fo\[G]gok.
    Kicsiny kis fényemmel \[H7]világítani fo\[Em]gok.
    Áldom \[D]Őt minden \[D7]nap és minden\[G]hol. \[C] \[G]
  \endverse

  \beginverse
    ^Elrejtsem-e fényemet? Nem! Világítani fogok.
    ^Elrejtsem-e fényemet? Nem! Világítani fo^gok.
    Elrejtsem-e fényemet? Nem! ^Világítani fo^gok.
    Áldom ^Őt minden ^nap és minden^hol. ^ ^
  \endverse

  \beginverse
    A ^Sátán sem állíthat meg! Nem! Világítani fogok.
    A ^Sátán sem állíthat meg! Nem! Világítani fo^gok.
    A Sátán sem állíthat meg! Nem! ^Világítani fo^gok.
    Áldom ^Őt minden ^nap és minden^hol. ^ ^
  \endverse

  \beginverse
    ^Így teszek, míg Jézus jön! Igen! Világítani fogok.
    ^Így teszek, míg Jézus jön! Igen! Világítani fo^gok.
    Így teszek, míg Jézus jön! Igen! ^Világítani fo^gok.
    Áldom ^Őt minden ^nap és minden^hol. ^ ^
  \endverse

\endsong

\beginsong{Király vagy}
[
  by={Matteo Mori, Csiszér László}
]

  \beginverse*
    /: \[D]Király vagy, \echo{Király vagy}
    \[A]Király v\_{a}gy, \echo{Király v\[Hm]{\_{a}}gy}
    Jézus Ki\[G]rály. :/ \rep{2}
  \endverse

  \beginverse*
    /: Most \[D]felemeljük a szívünket,
    Most \[A]felemeljük kezeinket,
    \[Hm]Trónod elé járul\_{u}nk, imádva \[G]légy! :/ \rep{2}
  \endverse

\endsong

\beginsong{Köszönöm Jézus - !!}[]

  \beginverse\memorize
    \[D] Köszönöm Jézus, köszönöm Jézus,
    Köszönöm U\[G]ram, hogy szeretsz en\[D]gem!
    Köszönöm \[F#m]Jézus, köszönöm \[Hm]Jé\[G]zus,
    Köszönöm U\[D]ram, \[A] hogy szeretsz en\[D]gem! \[G] \[D]
  \endverse

  \beginverse
    ^ A Golgotára ment, és meghalt értem
    Köszönöm U^ram, hogy szeretsz en^gem!
    A Golgo^tára ment, és meghalt ^ér^tem
    Köszönöm U^ram, ^ hogy szeretsz en^gem! ^ ^
  \endverse

  \beginverse
    ^ A harmadik napon, Ő feltámadott
    Köszönöm U^ram, hogy szeretsz en^gem!
    A harma^dik napon, Ő fel^táma^dott
    Köszönöm U^ram, ^ hogy szeretsz en^gem! ^ ^
  \endverse

  \beginverse
    ^ Ó alleluja, ó alleluja
    Köszönöm U^ram, hogy szeretsz en^gem!
    Ó alle^luja, ó alle^lu^ja
    Köszönöm U^ram, ^ hogy szeretsz en^gem! ^ ^
  \endverse

\endsong

\beginsong{Krisztus él bennem}[]

  \ifchorded
    \beginverse*
       {\nolyrics Előjáték: \[G] \[D] \[C] \[C-D]}
    \endverse
  \fi

  \beginchorus
    \[G]Krisztus él bennem, \[D]szívem virágzik.
    A \[Em]sötét nem győz, hát \[C]nem kell pánik.
    \[G]Jézus \[D]életét \[C]adta, hogy \[D]élj!

    \vspace{0.3cm}

    \[G]Krisztus él benned, \[D]ne légy lekvár,
    Hisz \[Em]oly sok kincs van, mi \[C]téged is vár!
    Hát \[G]kelj fel, \[D]indulj és \[C]Krisztussal \[D]járj!

    \vspace{0.3cm}

    /: Na-na-na, \[G]na na na \[D]na, na-na-na, \[Em]na na na \[C]na,
    na-na-na, \[G]na na na, \[D]na na na, \[C]na na! \[D] :/ \rep{2}
  \endchorus

  \beginverse\memorize
    Az \[Em]Örömhír fénye \[C]ragyogjon éle\[D]tedben!
    A \[G]szeretet tüze \[Am]égjen a szíved\[D]ben!
    Hogy \[Hm]lássa meg a vi\[Em]lág, te is \[C]mondd velünk, \[Am]éld meg és add to\[D]vább!
  \endverse

  \beginverse
    ^Hiába háborog ^alattunk a ten^ger,
    ^Jézus szemébe ^nézve félnünk ^nem kell.
    Ha ^Ő bennünk ^él, előttünk ^világos, ^tiszta a ^cél.
  \endverse

\endsong

\beginsong{Látod, újra este van}[]

  \beginverse\memorize
    \[C] Látod, újra este van, Hozzád száll most \[C/H]halk sza\[Am]vam,
    \[Dm] Mindenségnek Iste\[F]n\[G]{\_{e}}, Téged dicsér éne\[C]kem.
  \endverse

  \beginverse
    ^ Köszönöm ezt a napot, annyi szépet s ^jót ho^zott.
    ^ Köszönöm a napsü^t\_{é}^st, fényét szórta szerte^szét.
  \endverse

  \beginchorus
    \[C]Kérlek, ó jó Uram, \[Dm]add meg, fényeddé \[F]váljak, jeled le\[C]hessek!
    \[C]Kérlek, Szentl\_{e}lked \[Dm]küldd el, \_{Ő} ve\[F]zessen \[G]végig élete\[C]men!
  \endchorus

  \beginverse
    ^ Uram, mindent köszönök, bánatot é^s örö^möt,
    ^ Legyen rajtam szent ke^z\_{e}^d, óvja, védje éle^tem!
  \endverse

  \beginverse
    ^ Szeretném, ha szeretnél, minden szavad ^érte^ném,
    ^ Életemben követ^n{\_{é}}^m, s végül Hozzád érkez^nék.
  \endverse

  \beginverse*
    Refrén
  \endverse

  \beginverse*
    1. versszak \ifchorded{\nolyrics (Csak lehúzásokkal.)}\fi
  \endverse

\endsong

\beginsong{Lelkem, az Urat dicsérd}
[
  ititle={Az Úr dicsérete},
  sr={145. zsoltár alapján},
  by={Borka Zsolt},
  li={Kék könyv / 40.}
]

  \ifchorded
    \versesep=12pt plus 3pt minus 7pt
  \fi

  \ifchorded
    \beginverse*
       {\nolyrics Előjáték: \[F] \[C7] \[F] \[Gm] \[Am] \[A#] \[F] \[C] \[A#] \[F]}
    \endverse
  \fi

  \beginchorus
    /: \[F]Allelu\[Gm]ja, \[Am]allelu\[A#]ja, \[F]allelu\[C]ja, alle\[A#]lu\[F]ja! :/ \rep{2}
  \endchorus

  \beginverse\memorize
    \[F]Lelkem, az \[C]Urat di\[F]csérd, \[Gm]áldjad az \[C]istenek \[F]Isten\[C]ét!
    \[F]Jósága \[C]végtelen \[F]él, \[Gm]irgalma \[C]megmarad \[F] örök\[C]ké!
  \endverse

  \beginverse*
    Refrén
  \endverse

  \beginverse
    ^Minden szív ^áldjon, U^ram, ^magasztaljon ^Téged ^ég és ^Föld!
    ^Beszéljék ^hatalma^dat, ^zengjenek ^országod ^fényé^ről!
  \endverse

  \beginverse*
    Refrén
  \endverse

  \beginverse
    ^Nagyságod ^mérhetet^len, ^dicsőséged ^fenség, ^fényes^ség.
    ^Csodálva ^éneke^lem ^félelmes ^tetteid ^ ere^jét.
  \endverse

  \beginverse*
    Refrén
  \endverse

  \beginverse
    ^Közel vagy ^mindenki^hez, ki ^tiszta ^szívből ^Hozzád ki^ált,
    ^Kérését ^teljesí^ted, ^megőrzöd ^azt, ki ^Téged i^mád.
  \endverse

  \beginverse*
    Refrén
  \endverse

  \beginverse
    ^Minden nap ^dicsérem ^Őt, ^áldom az ^Urat, míg ^élek ^én.
    ^Új dallal ^színe e^lőtt, ^ujjongva ^hirdetem ^szent nev^ét!
  \endverse

  \beginverse*
    Refrén
  \endverse

\endsong

\beginsong{Loyolai Szent Ignác imája}
[
  index={Fogadd el}
]

  \ifchorded
    \beginverse*
       {\nolyrics Előjáték: \[C] \[G] \[Am] \[F] \[C] \[G] \[C] \[G7]}
    \endverse
  \fi

  \beginverse\memorize
    \[C]Fogadd el, \[G]Uram, \[Am]szabadságo\[F]mat, \[C]fogadd el \[G]egé\[C]szen! \[G]
    \[C] Vedd értel\[G]memet, \[Am]akarato\[F]mat, és \[C]emlé\[G]kezé\[C]sem!
  \endverse

  \beginverse
    ^Mindazt, amim ^van ^és ami ^vagyok, ^Te adtad ^in^gyen.^
    ^Visszaadok, ^Uram, ^visszaa^dok ^egyszerre ^min^dent.
  \endverse

  \beginchorus
    \[Am]Legyen \[E7]fölöttük \[Am]korlátlan \[F]úr \[C]rendel\[G]kezé\[C]sed!
    Csak \[Am]egyet \[E7]hagyj meg \[Am]ajándéko\[D7]dul: \[G]szeretnem \[F]Té\[Em]ged! \[G7]
  \endchorus

  \beginverse
    ^Csak a ^szeretet ^maradjon e^nyém a ^kegye^lem^mel, ^
    S ^minden, de ^minden ^gazdagság e^nyém, más ^semmi ^nem ^kell!
  \endverse

  \ifchorded
    \beginverse*
       {\nolyrics Utójáték: \[C] \[G] \[Am] \[F] \[C] \[G] \[C]}
    \endverse
  \fi

\endsong

\beginsong{A mélyből Hozzád száll szavam}[]

  \versesep=12pt plus 3pt minus 9pt

  \ifchorded
    \beginverse*
      {\nolyrics Előjáték: \[H] \[A] \[G] \[F#]}
    \endverse
  \fi

  \beginverse\memorize
    A \[Em]mélyből \[H7]Hozzád \[Em]száll sza\[Am]vam, \[G]Uram, \[H7]irgal\[Em]mazz!
    A \[Em]bajban \[H7]lelkem \[Em]társta\[Am]lan, \[G]Uram, \[H7]irgal\[Em]mazz!
  \endverse

  \beginchorus
    \[Em!]Segíts! \echo{Segíts!} \[Am!]Ne hagyj! \echo{Ne hagyj!}
    \[Em]Nézd, \[H7]ránk tör \[Em]a bűn már!
    \[Em!]Te adj \echo{Te adj} \[Am!]erőt! \echo{erőt!}
    \[G]Uram, \[H7]irgal\[Em]mazz!
  \endchorus

  \beginverse
    Ha ^ajkam ^vétett ^elle^ned, ^Krisztus, ^kegyel^mezz!
    A ^jóhírt ^zengjem ^csak Ve^led, ^Krisztus, ^kegyel^mezz!
  \endverse

  \beginchorus
    \[Em!]Segíts! \echo{Segíts!} \[Am!]Ne hagyj! \echo{Ne hagyj!}
    \[Em]Nézd, \[H7]ránk tör \[Em]a bűn már!
    \[Em!]Te adj \echo{Te adj} \[Am!]erőt! \echo{erőt!}
    \[G]Krisztus, \[H7]kegyel\[Em]mezz!
  \endchorus

  \beginverse
    Ha ^arcod ^fényét ^elfe^dem, ^Uram, ^irgal^mazz!
    Te ^újból ^hívjál, ^Iste^nem, ^Uram, ^irgal^mazz!
  \endverse

  \beginchorus
    \[Em!]Segíts! \echo{Segíts!} \[Am!]Ne hagyj! \echo{Ne hagyj!}
    \[Em]Nézd, \[H7]ránk tör \[Em]a bűn már!
    \[Em!]Te adj \echo{Te adj} \[Am!]erőt! \echo{erőt!}
    \[G]Uram, \[H7]irgal\[Em]mazz!
  \endchorus

\endsong

\beginsong{Mennek az asszonyok}
[
  by={Varga Attila},
  li={Kék könyv / 162.B}
]

  \ifchorded
    \beginverse*
      {\nolyrics Előjáték: /: \[D] \[D4-D-D2] :/ \rep{2}}
    \endverse
  \fi

  \beginverse\memorize
    \[D]Mennek az asszony\[A]ok, \[G]mennek az Úr \[D]sírjához,
    \[Em]Keresik \[D]azt, akit \[A]megöltek, \[Em]keresik \[D]azt, akit \[A]szerettek,
    \[G]Mennek az \[A]asszony\[D]ok.
  \endverse

  \beginverse
    ^Mennek az asszony^ok, ^mennek az Úr ^sírjához,
    ^Kezükben ^fehér ^gyolcs, ^szívükben ^fájda^lom,
    ^Mennek az ^asszony^ok.
  \endverse

  \beginchorus
    \[G]Miért keresitek Őt, az élőt a holtak között?
    \[D]Miért keresitek Őt? \[A]Nincsen itt, \[A7]feltáma\[D]dott! \[G]
    Nincsen \[D]itt, \[A] feltámad\[D]ott! \[D4] \[D]
  \endchorus

  \beginverse
    ^Jézus megmond^ta, ^harmadnapra ^feltámadt,
    ^Üres ^már a ^sír, ^terjed^jen a ^hír,
    ^Él a mi ^Megvál^tónk!
  \endverse

  \beginverse
    ^Mennek az asszony^ok, ^mi is megyünk ^utánuk,
    ^Szívünkben ^nagy ö^röm, ^ujjong^jon a ^Föld,
    ^Jézus, ^Ő az ^Úr!
  \endverse

  \beginverse*
    Refrén \rep{2}
  \endverse

\endsong

\beginsong{Minden mi él}[]
  \versesep=12pt plus 3pt minus 8pt

  \beginchorus
    \[D]Minden mi él, csak Téged hirdet,
    \[Hm]Minden dicsér, mert mind a műved,
    \[G]Azzal, hogy él, ezt zengi néked:
    \[A]Dicsérlek én, \[A7]dicsérlek Téged!
  \endchorus

  \beginverse\memorize
    \[D]Dicsér az ég, Nap, Hold és csillagok,
    \[Hm]Fény és sötét, nap, éj és hajnalok.
    \[G]Dicsér a szél, felhő és hóvihar,
    \[A]A víz, a tűz \[A7]megannyi tiszta dallal!
  \endverse

  \beginverse*
    Refrén
  \endverse

  \beginverse
    ^Dicsér a Föld, dicséri szent neved,
    ^Mint jó anyánk, táplál s ad eledelt.
    ^Virág, gyümölcs, zöld fű, fa, hegyvidék,
    ^Tó és folyó, ^síkság és büszke bérc!
  \endverse

  \beginverse*
    Refrén
  \endverse

  \beginverse
    ^A nagy világ létével énekel,
    ^Szavunkra vár, hogy hangja dal legyen.
    ^Zengjük tehát ég és föld énekét,
    ^Zengjük velük: ^nagy Isten áldott légy!
  \endverse

  \beginverse*
    Refrén
  \endverse

\endsong

\beginsong{Minden nap - !!}[]

  \ifchorded
    \beginverse*
      {\nolyrics Előjáték: \[C] \[G] \[Em] \[D]}
    \endverse
  \fi

  \beginverse\memorize
    \[C] Mit is \[G]mondhatnék, \[Em] Te vagy, ki \[D]megváltottál,
    \[C] Drága \[G]véreddel \[Em] hófe\[D]hérre mostál,
    \[C] Minden\[G]em lettél, \[Em] tiéd az \[D]egész lényem,
    \[C] Ég a \[G]tűz bennem, \[Em] csak \[D]érted élek már!
  \endverse

  \ifchorded
    \beginverse*
      {\nolyrics Közjáték: \[G] \[C] \[Em] \[D]}
    \endverse
  \fi

% közjáték -> utána ugyanaz

  \beginverse
    ^ Minden ^nap, Uram, ^ ráál^lok igédre,
    ^ Téged ^hívlak, ^ légy a ^segítségem,
    ^ Vezess ^át engem, ^ tovább a ^keskeny úton,
    ^ Add, hogy ^életem ^ a világban ^tüzet gyújtson!
  \endverse

  \beginchorus
    /: \[G] Minden \[C]nap csak \[Em]érted \[D]élek.
    \[G] Minden \[C]nap kö\[Em]vetlek \[D]Téged.
    \[G] Minden \[C]nap Ve\[Em]led já\[D]rom u\[G]tam. \[C] \[Em] \[D] :/ \rep{2}
  \endchorus

  \beginverse
    /: \[G]Érted \[C]élek \[Em]minden \[D]nap. :/ \rep{3}
    U\[G]ram \[C] \[Em] \[D]
  \endverse

\endsong

\beginsong{Mint az asszony}[]

  \beginverse*\memorize
    Mint az \[A]asszony az Ő \[E/G#]ruhája szegé\[F#m]lyét, \[F#m/E]
    Megragad\[D]juk \[D/C#] az Úr je\[Hm]lenl{\_{é}}\[E]tét.
    Mint a \[A]vak koldus az \[E/G#]{\_{ú}}t sz\_{é}\[F#m]lén, \[F#m/E]
    Kiál\[D]tunk, \[D/C#] s fu\[Hm]tunk az Úr e\[E]lé.
  \endverse

  \beginchorus
    És \[A]hirtelen \[E] egy mennyei \[D]kéz megáld, \[D/C#]
    \[Hm]Jézus az, \[E] aki megse\[A]gít! \[E]
    És \[A]hirtelen \[E] a mennyei \[D]erő leszáll, \[D/C#]
    \[Hm]Jézus jön \[E] s meggy\_{ó}\[A]gyít!
  \endchorus

\endsong

\beginsong{Most nem sietek}
[
  by={Túrmezei Erzsébet}
]

  \beginverse\memorize
    \[Em]Most nem sie\[G]tek, most nem roha\[D]nok,
    \[Em]Most nem terve\[G]zek, \[C]most nem aka\[D]rok,
  \endverse

  \beginchorus
    \[G]Most nem teszek \[D]semmit sem,
    Csak \[C]engedem, hogy \[D]szeressen az \[G]Ist\_{e}n.
    \[G]Most nem teszek \[D]semmit sem,
    \[G]Csak engedem, hogy \[H7]szeressen az \[Em]Isten.
  \endchorus

  \beginverse
    ^Most m\_{e}gnyug^szom, most elpihe^nek,
    ^Békén, szaba^don, ^mint gyenge gye^rek,
    Nem, nem teszek ...
  \endverse

  \beginverse
    ^S míg ölel a ^fény, és ölel a ^csend,
    ^És árad be^lém, ^és újjáte^remt,
    Míg nem teszek ...
  \endverse

  \beginverse
    ^Új gyümölcs te^rem, másoknak te^rem,
    ^Érik csende^sen ^erő, győze^lem,
    Ha nem teszek ...
  \endverse

\endsong

\beginsong{Mustármag - !!}[]

  \beginverse
    /: Hogyha csak \[Am]mustármagnyi hitetek is \[E]volna... -
    - ezt mondotta a mi U\[Am]runk. :/ \rep{2}
  \endverse

  \beginverse
    /: És ha így \[Dm]szóltok a nagy he\[Am]gyekhez:
    Mozdulja\[E]tok, mozdulja\[Am]tok! :/ \rep{2}
  \endverse

  \beginverse
    /: \[Am] Akkor a hegyek megindul\[E]nak,
    Megindulnak, megindul\[Am]nak. :/ \rep{4}
  \endverse

  \beginverse
    /: \[Am]Jöjj el, jöjj el, ó \[E]jöjj el Szent\[Am]lélek! :/ \rep{4}
  \endverse

\endsong

\beginsong{Napfivér, Holdnővér}[]

  \ifchorded
    \beginverse*
      {\nolyrics Előjáték: \[D4] \[D] \[D4] \[D]}
    \endverse
  \fi

  \beginverse\memorize
    \[D]Fivérem \[F#m]Nap és \[G]n{\_{ő}}\[A]vérem \[D]Hold,
    \[Hm]Oly' ritkán \[F#m]látlak, s \[G]hallom \[A]hango\[D]tok.
    \[Hm] Nyomaszt a \[F#m]sok \[G]gyötre\[A]lem és \[D]gond.
  \endverse

  \ifchorded
    \beginverse*
      {\nolyrics Közjáték: \[Hm] \[F#m] \[G] \[A] \[D]}
    \endverse
  \fi

  \beginverse
    ^Fivérem ^Szél és ^Le^veg\_{ő}^ég,
    ^ Nyisd ki sze^mem, hogy ^lássam, ^ami ^szép!
    ^ Körülö^lel a ^ragyo^gás, dics^fény.
  \endverse

  \beginchorus
    \[Hm]Mert Isten \[F#m]m{\_{ű}}ve \[G]minden \[A]teremt\[D]mény,
    \[Hm] Érzem jó\[F#m]ságát, és \[G]szívem újra {\[A4 A]é}l!
  \endchorus

  \beginverse
    ^Fivérem ^Nap és ^n{\_{ő}}^vérem ^Hold,
    ^Most végre ^látok, s ^hallom ^hango^tok!
    ^ Megölel^ném az ^egész ^vil\_{á}^got!
  \endverse

  \beginverse*
    Refrén
  \endverse

  \beginverse*
    ^Fivérem ^Nap és ^n{\_{ő}}^vérem ^Hold,
    ^Most végre ^látok, s ^hallom ^hango^tok!
    ^ Megölel^ném az ^egész ^vil\_{á}\[Gm]got! \[Gm7] ^
  \endverse

\endsong

\beginsong{Ne félj, mert megváltottalak -- !!}[]

  \beginchorus
    /: \[D]Ne félj, mert \[G]megváltotta\[A]lak,
    Neveden \[D]szólította\[Hm]lak,
    Karja\[Em]imba zárta\[A]lak,
    Örökre \[A7]enyém \[D]vagy! :/ \rep{2}
  \endchorus
  
  \beginverse\memorize
    \[A]Viruló réteken \[D]át, \[G]hűs forrás fe\[A]lé vezetlek,
    \[Hm]Pásztorod va\[Em]gyok, elveszni \[A]senkit nem ha\[D]gyok.
    Karom fe\[Hm]léd tá\[Em]rom, kiárad \[G]áldá\[D]som.
    \[Hm]Nem rejtőzőm \[Em]el, szívem a \[A]szívednek fe\[D]lel,
    Amikor \[Hm]úgy ér\[Em]zed, nyomaszt a \[A7]é\[D]let.
  \endverse

  \beginverse
    ^Nem taszítalak ^el, ^amikor vétke^zel,
    Irgalmat ^lelsz a szívem^ben, örök fe^léd a hűsé^gem,
    Amerre ^jársz, vé^dlek, nyomodba ^lé^pek.
    ^Nem rejtőzöm ^el, szeretet^lángom átö^lel,
    Ne félj, ha ^éjben ^jársz, hidd, hogy a ^fény vár ^rád!
  \endverse

\endsong
\beginsong{Ne félj, ne aggódj}[]

  \beginverse*
    \[Am]Ne félj, ne \[Dm7]aggódj, \[G]ne sírj, ne \[Cmaj7]bánkódj:
    \[F]Ha tiéd \[Dm6]Isten, \[E]tiéd már \[Am]minden.
    \[Am]Ne félj, ne \[Dm7]aggódj, \[G]ne sírj, ne \[Cmaj7]bánkódj:
    \[F]Elég \[Dm6]Ő \[E]né\[Am]ked.
  \endverse

\endsong

\beginsong{Nézz, testvér, fel - !!}[]

  \beginchorus
    \[D]Nézz, testvér, fel, az \[F#m]Úr van itt, \[G]lángol\[A]jon a szí\[D]vünk!
    \[G]Közel van már \[F#m]üdvös\[Hm]ségünk,
    \[G]Jöjj el \[Em]Jézu\[A]sunk, \[G]jöjj el \[A]Jézu\[D]sunk!
  \endchorus

  \beginverse\memorize
    \[D]Kerestem arcodat, U\[F#m]ram, leha\[G]joltál \[A]énhoz\[D]zám.
    \[D]Félelmek, bűnök tép\[F#m]tek, de Te \[G]meggyó\[A]gyítot\[D]tál.
  \endverse

  \beginverse*
    Refrén
  \endverse

  \beginverse
    ^Figyelme szeretőn kí^sér, meghall^gatja ^az i^mám.
    ^Gyötrelmeimben vigaszt nyújt, ezért ^áldja ^Őt a ^szám!
  \endverse

  \beginverse*
    Refrén
  \endverse

  \beginverse
    ^Jöjjön, ki szomjas és i^gyék, aki ^hisz, ^élni ^fog!
    ^Szívének rejtett mélyé^ből az é^lő víz ^felbu^zog.
  \endverse

  \beginverse*
    Refrén
  \endverse

\endsong

\beginsong{Néked hódolok}[]

  \beginverse*
    \[A]Néked hódo\[Hm]lok, Jé\[E]zus, dicső Ki\[A]rály! \[D] \[E]
    \[A]Néked hódo\[Hm]lok, Jé\[E]zus, dicső Ki\[A]rály!
    Ujjongj, da\[Hm]lolj szív\[E]{\_{e}}m, és imádjad \[A]Őt! \[F#m]
    Ujjongj, da\[Hm]lolj szív\[E]{\_{e}}m, és imádjad \[A]Őt!
  \endverse

\endsong

\beginsong{Nincs más isten}
[
  by={Chris Tomlin, fordította Váradi Attila},
  li={http://nyugodtleszazeleted.blogspot.hu/2010/11/nincs-mas-isten.html},
  index={Nálad lett borrá a víz}
]

  \beginverse\memorize
    \[Em]Nálad lett \[C]borrá a \[Em]víz, \[C] h\[Em]oltakat \[C]életre \[G]hívsz,
    Nincs más \[Am7]isten, nincsen \[D]más.
  \endverse

  \beginverse
    ^Te ragyo^god be az ^éjt, ^ ^Te hozod ^el a re^ményt.
    Nincs más ^isten, nincsen ^más.
  \endverse

  \beginchorus
    \[Em]Te vagy az Isten, \[C]aki nevére \[G]leborul minden a \[D]Földön, az égen,
    \[Em]Hegyeket mozdít, \[C]beteget gyógyít az \[G]Úr ma i\[D]s!

    \[Em]Te vagy az Isten, \[C]aki nevére \[G]leborul minden a \[D]Földön, az égen,
    \[Em]Szíveket hódít, \[C]sebeket gyógyít az \[G]Úr ma i\[D]s!

    /: \[Em] És hogyha Isten velünk, \[C] ember mit árthat nekünk,
    \[G] Mert hogyha Te vagy velünk, \[D] ki lehet ellenünk? :/ \rep{2}
  \endchorus

  \beginverse*
    Refrén \echo{\[D]Mi állhat ellenünk?}
  \endverse

\endsong

\beginsong{Oly' jó áldani}
[
  sr={92. zsoltár alapján},
  by={Ferenczy Rudolf (Dax), Rozsályi Z.}
]

  \beginverse*
    \[F]Szívemből \[C]ünnepi \[F]ének á\[C]rad,
    \[G]Háladalt \[G7]éneklek az \[F]ég Urá\[C]nak!
    \[G]Ujjongva \[G7]tör fel a \[F]hála ben\[C]nem,
    \[Dm]Jósága körülvesz \[G]szárnyként \[G7]en\[C]gem. \[Dm] \[G] \[C]
  \endverse

  \beginverse*
    /: \[G]Adjatok \[G7]hálát az \[C]Istennek!
    \[G7]Adjatok hálát az \[C]Istennek!
    \[C7]Ő a mi teremtő \[F]Kirá\[G]lyunk,
    \[D]Alle-allelu\[G]ja, \[F]allelu\[C]ja, \[F]allelu\[C]ja! :/ \rep{2} \[H7]
  \endverse

  \beginverse*
    /: \[E]Hála legyen, hála legyen, hála! \[A]Áldott az \[E]Úr! :/ \rep{3}
  \endverse

  \ifchorded
    \beginverse*
      {\nolyrics Átvezető: \[D] \[A] /: \[A] \[E] \[A] \[E] :/ \rep{2}}
    \endverse
  \fi

  \beginverse\memorize
    \[A]Oly' jó ál\[E]dan\[A]i az \[E]Ur\[A]at, \[A] \[E] \[A] \[E]
    \[E]Ujjongó\[A]an \[E]jó Is\[A]ten\[E]nek zen\[A]ge\[D]ni zsoltáro\[A]kat. \[A] \[E] \[A] \[E]
  \endverse

  \beginverse
    ^Már kora reg^ge^len jó^sá^gát, ^ ^ ^ ^
    És ^éjszaká^kon ^át hir^de^tem szün^te^len nagy irgal^mát. ^ ^ ^ ^
  \endverse

  \beginverse
    ^Pengő hang^sze^ren, lan^to^kon, ^ ^ ^ ^
    ^Ujjongó^an ^jó Is^ten^nek zen^ge^ni hála^szót. ^ ^ ^ ^
  \endverse

  \beginverse
    ^Az öröm hang^ja^i szól^ja^nak, ^ ^ ^ ^
    ^Gitár húr^ja^in magasz^ta^lom uj^jong^va csodái^dat! ^ ^ ^ ^
  \endverse

  % \ifchorded
  %   \chordsoff
  % \fi

    \beginverse
      ^Uram, jó voltál énhozzám,
      Életemben nagy dolgokat műveltél sok év során.
    \endverse

    \beginverse
      Sziklám, oltalmam Te voltál,
      Oly' irgalmasan jó, gondoskodó Istenem, égi Atyám.
    \endverse

  % \ifchorded
  %   \chordson
  % \fi

  \beginverse
    ^Hála, di^cső^ség ne^ved^nek, ^ ^ ^ ^
    ^Most és ö^rök^ké az ^é^gen, a ^Föl^dön! \_{A}^men. ^ ^ ^ ^ \[A]
  \endverse

\endsong

\beginsong{Ott lélegzel a fákban}[]

  \ifchorded
    \beginverse*
       {\nolyrics Előjáték: \[D] \[A] \[A7] \[D]}
    \endverse
  \fi

  \beginverse\memorize
    \[D]Ott lélegzel a \[A]f{\_{á}}k\[D]ban, \[G]ott mosolyogsz a vi\[A]r{\_{á}}gban,
    \[A]Ott vagy a szélben, a \[D]fr{\_{i}}ssben, ó, boldog, b\_{o}ldog \[A7]{\_{I}}s\[D]ten!
  \endverse

  \beginverse
    ^Szép sorsomat Te ^sz{\_{ő}}^tted, ^jó leborulni e^l{\_{ő}}tted,
    ^Te vagy az útnak a ^v{\_{é}}ge, sz\_{e}nt mélység, \_{á}ldott ^b{\_{é}}^ke!
  \endverse

\endsong

\beginsong{Ő az Úr -- !}[]

  \beginverse\memorize
    Ő az \[H7]Úr! Ő az \[E]Úr! Meghalt értem a ke\[F#m?]reszten, Ő az \[H7]Úr!
  \endverse

  \beginchorus
    Minden \[E]térd megha\[E7]joljon,
    Minden \[A]nyelv csak Róla \[F#m]szóljon,
    Mert \[E]Jézus, \[H7]Ő az \[E A E]Úr!
  \endchorus

  \beginverse
    Ő az ^Úr! Ő az ^Úr! Feltámadott a ha^lálból, Ő az ^Úr!
  \endverse

  \beginverse
    Ő az ^Úr! Ő az ^Úr! Betölt engem Szentlel^kével, Ő az ^Úr!
  \endverse

\endsong

\beginsong{Péter, ne sírj}[]

  \beginverse\memorize
    \[Em]Sírásodat hagyd nyugodtan abba,
    A kapuk mögött vissza\[Em/F#]állt a \[G]rend!
    \[Am]Ne sírj, oly' lágyan jár a \[Em]szél
    \[C]Róma s a \[D]Via Appia fe\[Em]lett!
  \endverse

  \beginchorus
    \[Am]Péter, ne \[Em]sírj, mert \[H7]Jézus megbo\[Em]csát,
    \[Am]Péter, ne \[Em]sírj, hisz' \[H7]tudod, hogy megbo\[Em]csát!
  \endchorus

  \beginverse
    ^Gyengeségünk néha úgy esik ránk,
    Mint kisgyermekre esik a ^féle^lem.
    ^Gyengeségért mindig bünte^tés jár,
    De ^büntetése ^nem sújt, fele^mel!
  \endverse

  \beginverse*
    Refrén
  \endverse

  \beginverse
    ^Reménytelen a bűn után az élet,
    Hogy mégis élsz, már alig ^hiszed ^el,
    ^Alig hiszed, hogy nem lett akkor ^vége,
    Csak ^egyet tudsz, a ^hulló könnye^ket!
  \endverse

  \beginverse*
    Refrén \rep{2}
  \endverse

\endsong

\beginsong{Rajta, dicsérjétek az Úr nevét - !!}[]

  \ifchorded
    \beginverse*
      {\nolyrics Előjáték: \[G-C] \[G!]}
    \endverse
  \fi

  \beginchorus
    Rajta, di\[G]csérjé\[C]tek az Úr ne\[G]vét,
    Ti \[C]mindnyájan, kik \[G]hűen \[Em]szolgáljátok \[D]Őt! \[D4] \[D]
    Áldjátok \[G]hát há\[C]romszor szent ne\[G]vét,
    Ma\[C]gasztaljátok \[G]ének\[Am]szóval \[D]örök\[G]ké! \[C] \[G]
  \endchorus

  \beginverse\memorize
    Igen, tu\[Em]dom, Istenünk hatal\[C]mas,
    Végbevisz \[Am]mindent, amit csak a\[D]k\[H7]ar.
    A hegyek\[Em]től a tenger mélyé\[C]ig,
    A föld\[Am]től az égbolt magasságá\[D]ig.
  \endverse

  \beginverse*
    Refrén
  \endverse

  \beginverse
    Lássátok ^hát, mily’ irgalmas hoz^zánk,
    Jóságos ^Úr és hűséges Ki^r^ály.
    Énekel^jétek hát az Ő ne^vét,
    Istent di^csérje minden teremt^mény!
  \endverse

  \beginverse*
    Refrén \rep{2}
  \endverse

\endsong

\beginsong{Rejts most el}
[
  by={Reuben Morgan, fordította Labadics Kriszta}
]

  \beginverse\memorize
    \[C]Rejts \[G/H]most \[Am]el a \[F]szár\[D/F#]nyad a\[G]lá,
    E\[C]rős \[G]kéz\[F]zel \[Dm]takarj \[D]be en\[G]gem!
  \endverse

  \beginchorus
    \[C]Tenger tombol, \[F]zúg, süv\[G]ít a sz\[C]él,
    \[C]Te emelsz fel \[F]a vih\[G]ar fö\[Am]lé!
    \[C]Uralkodsz hul\[F]lámok h\[G]abja\[C]in,
    \[C]Szívem nem f\[F]él, Benn\[G]ed rem\[Am]él!
    \[C]Szívem nem f\[F]él, Benn\[G]ed rem\[F]él! \[G]
  \endchorus

  \beginverse
    Csak ^Is^ten^ben bízz, ^én ^lel^kem!
    ^Mert ^Ő ^él, ^nagyobb ^minden^nél!
  \endverse

  \beginverse*
    Refrén  \[C]
  \endverse

\endsong

\beginsong{Szentlélek, jöjj}[]

  \versesep=12pt plus 3pt minus 12pt

  \ifchorded
    \beginverse*
       {\nolyrics Előjáték: \[Em] \[D] \[Hm] \[Em] \[D] \[Hm] \[Em]}
    \endverse
  \fi

  \beginchorus\memorize
    \[Em]Szentlélek, jöjj, lobogó \[D]Láng!
    Szentlélek, jöjj, \[Hm]a világ \[Em]vár!
    Szentlélek, jöjj, viharos \[D]szél!
    Jöjj, \[Hm]áradj \[Em]szét!
  \endchorus

  \beginverse
    ^Jöjj el, Élővíz for^rás,
    Jöjj, a szívünk ^Téged ^vár!
    Jöjj, ki fényt adsz lelkünk^nek,
    Jöjj, úgy ^várunk ^Rád!
  \endverse

  \beginverse*
    Refrén
  \endverse

  \beginverse
    ^Jöjj, igazság forrá^sa,
    Jöjj, imádunk ^mindnyá^jan!
    Jöjj, reményünk éleszd ^fel,
    Jöjj ke^gyelmed^del!
  \endverse

  \beginverse*
    Refrén
  \endverse

  \beginverse
    ^Jöjj, a néped gyűjtsd egy^be!
    Jöjj, az alvót ^ébreszd ^fel!
    Jöjj, a bűntől tisztíts ^meg,
    Báto^ríts min^ket!
  \endverse

  \beginverse*
    Refrén \rep{2}
  \endverse

\endsong

\beginsong{Szeretem örökké}
[
  index={Eléd lépek, jó uram}
]

  \beginverse\memorize
    \[C]Eléd lépek, jó Uram, hol \[G/H]béke és nyugalom \[Am]vár,
    \[Em]Vágyom, hogy érints \[Dm]Lelkeddel, hogy \[F]múljék, ami \[G]fáj.
    Én \[C]hiszem, hogy itt vagy közöttünk és \[G/H]halkan, szelíden \[Am]hívsz.
    És mi \[Em]társaid leszünk \[Dm]örökké, hol \[F]boldog dalát \[G]zengi minden \[C]szív.
  \endverse

  \beginchorus
    /: \[Am]Áldom az Urat míg \[Em]élek, és \[F]szeretem örök\[C]ké,
    \[Am]Örömmel mondok \[Em]hálát a \[F]világon minden\[G]ért. :/ \rep{2}
  \endchorus

  \beginverse
    ^Közénk térdelsz, jó Urunk, ^féltő szeretet^tel,
    És egy ^új életnek ^boldogsága ^érző szívre ^lel.
    Én ^tudom, hogy itt vagy közöttünk és ^halkan, szelíden ^hívsz.
    És mi ^társaid leszünk ^örökké, hol ^boldog dalát ^zengi minden ^szív.
  \endverse

  \beginverse*
    Refrén
  \endverse

\endsong

\beginsong{Táncolj az Úrnak}
[
  li={Kék könyv / 201.}
]

  \ifchorded
    \versesep=12pt plus 3pt minus 9pt
  \fi

  \ifchorded
    \beginverse*
      {\nolyrics Előjáték: /: \[Em] \[D] \[Em] \[Hm] \[G] \[D] \[Em] \[Hm] \[Em] :/ \rep{2}}
    \endverse
  \fi

  \beginchorus
    /: \[Em]Táncolj az \[D]Úrnak, \[Em]dicsérjed ne\[Hm]vét!
    \[G]Ujjongj, hisz' \[D]Ő a te \[Em]Meg\[Hm]vál\[Em]tód! :/ \rep{2}
  \endchorus

  \beginverse\memorize
    Mint \[Em]Dávid az \[D]Úr lá\[Em]dája e\[Hm]lőtt,
    \[G]Táncol\[D]junk \[Em]Is\[Hm]ten\[Em]nek!
    Az \[Em]Ő or\[D]szága már \[Em]köztünk \[Hm]van,
    \[G]Daloljon \[D]lelkünk \[Em]uj\[Hm]jong\[Em]va!
  \endverse

  \beginverse*
    Refrén
  \endverse

  \beginverse
    Mint ^Mária ^Erzsébet ^házá^ban,
    ^Ujjong a ^bensőm ^há^lá^val.
    A ^Lélek a ^szívemet ^eltöl^ti,
    ^Alle^luja, ^így ^zen^gi!
  \endverse

  \beginverse*
    Refrén
  \endverse

  \beginverse
    Kik ^kedvesek az ^Úrnak, ^dicsérik ^Őt,
    ^Hűsé^gükkel ^szol^gál^nak.
    Mint ^betlehemi ^nyájak ^pásztora^i,
    ^Hódol^junk ^szí^ne e^lőtt!
  \endverse

  \beginverse*
    Refrén
  \endverse

  \ifchorded
    \beginverse*
      {\nolyrics Utójáték: /: \[Em] \[D] \[Em] \[Hm] \[G] \[D] \[Em] \[Hm] \[Em] :/ \rep{2}}
    \endverse
  \fi

\endsong

\beginsong{Terád vár egy szép ország}[]

  \beginverse\memorize
    Terád \[E]vár egy szép ország,
    Terád \[A]vár egy szép or\[E]szág,
    Terád vár egy szép ország, ahová me\[H7]gyek, ahová megyek.
    Terád \[E]vár egy szép or\[E7]szág,
    Terád \[A]vár egy szép or\[Am]szág,
    Terád \[E]vár egy szép or\[H7]szág, ahová me\[E]gyek, aho\[A]vá me\[E]gyek.
  \endverse

  \beginverse
    Nincs ott ^többé könnyezés,
    Nincs ott ^többé könnye^zés,
    Nincs ott többé könnyezés, ahová me^gyek, ahová megyek.
    Nincs ott ^többé könnye^zés,
    Nincs ott ^többé könnye^zés,
    Nincs ott ^többé könnye^zés, ahová me^gyek, aho^vá me^gyek.
  \endverse

  \beginverse
    Isten ^szép országa ez,
    Isten ^szép országa ^ez,
    Isten szép országa ez, ahová me^gyek, ahová megyek.
    Isten ^szép országa ^ez,
    Isten ^szép országa ^ez,
    Isten ^szép országa ^ez, ahová me^gyek, aho^vá me^gyek.
  \endverse

  \beginverse
    Jézus ^Krisztus vár ott rád,
    Jézus ^Krisztus vár ott ^rád,
    Jézus Krisztus vár ott rád, ahová me^gyek, ahová megyek.
    Jézus ^Krisztus vár ott ^rád,
    Jézus ^Krisztus vár ott ^rád,
    Jézus ^Krisztus vár ott ^rád, ahová me^gyek, aho^vá me^gyek.
  \endverse

  \beginverse
    Add át ^néki az életed,
    Add át ^néki az éle^ted,
    Akkor Isten szép országát elnye^red, te is elnyered.
    Add át ^néki az éle^ted,
    Add át ^néki az éle^ted,
    Akkor ^Isten szép or^szágát elnye^red, te is ^elnye^red. \[A] \[E]
  \endverse

\endsong

\beginsong{Teremts bennem}
[
  li={Kék könyv / 208.}
]

  \beginverse*
    /: \[G]Teremts bennem \[D]tiszt\_{a} szí\[C]vet, \_{ó} U\[G]ram!
    Az \[Em]erős lelket \[D]újítsd meg ben\[G-C]nem! \[G] :/ \rep{2}
  \endverse

  \beginverse*
    /: \[Am]Ne vess el en\[D]gem a Te or\[G]cád e\[E]lől!
    \[Am]Szentlelked \[D]ne vondd meg tő\[G]lem! \[G7]
    \[Am]Támogass az \[D]engedelmes\[G]ség lelké\[Em]vel!
    \[Am7]Szabadításod \[D7]örömét add ne\[G]kem! \[G7] :/ \rep{2}
  \endverse

  \beginverse*
    \[Am7]Szabadításod \[D7]örömét add ne\[G-C]kem! \[G]
  \endverse

\endsong

\beginsong{Uram, Tehozzád futok}[]

  \beginverse*\memorize
    Uram, \[Em]Tehozzád fu\[D]tok, élő \[Hm7]vízre szomja\[Em]zom,
    Közel\[C]séged, ó mily \[D]jó énné\[Em]kem. \[Em!]
  \endverse

  \beginverse*
    Kérlek, ^ne menj el tő^lem, légy min^dig segítsé^gem,
    Úgy kí^vánlak Té^ged, Iste^nem. ^
  \endverse

  \beginverse*
    Én pedig \[G]szüntelen remél\[D]ek, egyre \[Hm7]jobban dicsér\[Em]lek,
    Ajkam \[Am7]beszéli a Te \[Am/F#]igazságo\[H4]dat. \[H]
    Hadd le\[G]gyen most a da\[D]lom jó i\[Hm]llat oltáro\[Em]don,
    Nagyon \[C]szeretlek Té\[D]ged, Jézu\[Em]som. \[Em!]
  \endverse

  \beginverse*
    \textnote{A legvégén az utolsó sor helyett:}
    Nagyon \[C]szeretlek Té\[D]ged,
    Nagyon \[C]szeretlek Té\[D]ged,
    Nagyon \[C]szeretlek Té\[D]ged, Jézu\[Em]som. \[Em!]
  \endverse

\endsong

\beginsong{Az Úrra vár a szivünk}[]

  \beginverse*
    Az \[Dm]Úrra \[C]vár a szi\[F]vünk,
    Ő\[Gm]benne \[Dm]minden \[A]örö\[Dm]münk.
  \endverse

  \ifchorded
    \beginverse*
      {\nolyrics ( Köztes: \[Dm] \[A#] \[C]
      \[F] \[Dm] \[Em] \[A] )}
    \endverse
  \fi

\endsong

\beginsong{Utad vár rád}[]

  \ifchorded
    \beginverse*
      {\nolyrics Előjáték: \[Dm-Dm4-Dm] \[F-F2-F] \[C-C4-C] \[Dm!-A!]}
    \endverse
  \fi

  \beginverse\memorize
    \[Dm]Mélyen a \[A]lelkedben egy \[Dm]tiszta fo\[C]lyó,
    Amit \[F]áradni \[Gm]érzel, Ő a \[C]Mindenha\[Dm]tó.
    \[Dm]Isteni \[A]üzenet az \[Dm]éjben a \[C]fény,
    \[F]Királysága \[Gm]már a \[C]szívedben \[Dm]él!
  \endverse

  \beginchorus
    Utad \[F]vár \[C]rád, \[Gm]indulj \[Dm]hát!
    \[F]Nálad van a \[Gm]kincs, mire \[C]mindenki \[Dm]vágy,
    Utad \[F]vár \[C]rád, \[Gm]indulj \[Dm]hát!
    \[F]Benned az \[G]isteni \[Gm]tűz a földre \[Dm]száll.
  \endchorus

  \beginverse
    ^Reménységed ^forrása ^égi A^tyád,
    ^Szeretetének ^nincs, ki ^ellen^áll.
    ^Te lehetsz a ^fénylő jel a ^hegy tete^jén,
    ^Te lehetsz, ki ^elviszed az ^üzene^tét.
  \endverse

  \beginverse*
    Refrén \rep{2}
  \endverse

\endsong

\beginsong{A vak ember - !!}[]

  \versesep=12pt plus 3pt minus 8pt

  \ifchorded
    \beginverse*
      {\nolyrics Előjáték: \[Am] \[G] \[F] \[E]}
    \endverse
  \fi

  \beginverse\memorize
    Az \[Am]úton a \[G]vak ember \[F]ül és ki\[E]ált, \rep{3}
  \endverse

  \beginchorus
    Oh, oh, \[E7]oh!
    Hol van az {\[Am G F]ú}t? \[E]Hol van a f{\[Am G F]é}ny, \[E]az igazs{\[Am G F]á}g,
    Ami haza\[E]visz? Oh, \[E7]oh, oh, \[E]oh, oh, \[E7]oh, oh, oh! (- Egyel kevesebb oh?)
  \endchorus

  \beginverse
    A ^béna is ^az úton ^ül és ki^ált, \rep{3}
  \endverse

  \beginverse
    A ^néma is ^az úton ^ül és tátog, \rep{2}
    A \[Am]néma is \[G]az úton \[F]ül és \[E]kiált: -- ez itt így jó?
  \endverse

  \beginverse
    S ?mi? \[Am]mind csak ü\[G]lünk, és a \[F]szívünk ki\[E]ált, \rep{3} -- kell ide "mi"?
  \endverse

  \beginverse
    De \[Am]Jézus is \[G]ott ül ve\[F]lünk és ki\[E]ált, \rep{3}
    Oh, oh, \[E7]oh!
    Én vagyok az {\[Am G F]ú}t, az igazs{\[Am G F]á}g, és az élet,
    Ami haza\[E]visz! Oh, \[E7]oh, oh, \[E]oh, oh, \[E7]oh, oh, oh! (- Egyel kevesebb oh?)
  \endverse

  \beginverse*
    \[Am]Jézus \rep{2}
  \endverse

  \beginverse*
    Oh, oh, \[E7]oh!
    Én vagyok az {\[Am G F]ú}t, az igazs{\[Am G F]á}g, és az élet,
    Ami haza\[E]visz! Oh, \[E7]oh, oh, \[E]oh, oh, \[E7]oh, oh, oh! (- Egyel kevesebb oh?)
  \endverse

  \beginverse*
    \[Am]Jézus
  \endverse

\endsong

\beginsong{Várj és ne félj}[]

  \beginverse*
    \[Em]Várj és ne \[C]félj, az \[Am6]Úr jön \[H7]már!
    \[Em]Várj \[D]és ne \[G]félj, hű \[Am7]szív\[H]vel \[Em]várj!
  \endverse

\endsong

\beginsong{A világnak Krisztus kell}[]

  \ifchorded
    \beginverse*
      {\nolyrics Előjáték: \[E] \[H7] \[E] \[A] \[E]}
    \endverse
  \fi

  \beginchorus
    /: A \[E]világnak \[H7] Krisztus \[E]kell!
    A \[A]világnak \[E] Krisztus \[H7]kell!
    A \[E]világnak \[G#]kellesz \[C#m]te is,
    \[A]mivel te \[E]Krisztushoz \[H7]tartoz\[E]ol! :/ \rep{2}
  \endchorus

  \beginverse*
    \[A]Emlé\[E]kezz! \[A]Emlé\[E]kezz, te \[E]Krisztushoz \[H7]tartoz\[E]ol.
  \endverse

\endsong

\beginsong{Virrassz még}
[
  by={Dr. Király Péter}
]

  \ifchorded
    \beginverse*
      {\nolyrics Előjáték: \[Dm] \[C] \[Dm] \[C] \[A#] \[A] \[A#] \[A]}
    \endverse
  \fi

  \beginverse*\memorize
    \[Dm] Gyermekét ébren őrző,
    \[C] Vak éjtől mit sem félő \[A#]asszony, Mári\[A]a!
    \[Dm] Jászolban mélyen alvó
    \[C]Kisdede fölé hajló \[A#]asszony, Mári\[A]a!
    Ki \[Dm]két karodban \[C]ringatgattad, \[F] hideg széltől \[A#]eltakartad \[F]Őt:
    \[Gm]Ég és Föld U\[A]rát;
    S ki \[Dm]félted, óvod \[C]minden bajtól \[F] azóta is \[A#]minden asszony \[F]védtelen,
    \[Gm]Apró magza\[A]tát;
  \endverse

  \beginchorus
    /: Virrassz \[Dm]még, \[C]Mária, virrassz \[F]még!
    Még \[Gm]vaksötét az \[C]éj, óvd a \[F]gyermek éle\[A]tét!
    Virrassz \[Dm]még, \[C]Mária, virrassz \[F]még,
    Míg \[Gm]fényét küldi \[A]újra ránk az \[Dm]ég! :/ \rep{2}
  \endchorus

\endsong

\beginsong{Vizek felett \\ Oceans}
[
  by={Hillsong United}
]

  \ifchorded
    \beginverse*
      {\nolyrics Előjáték: /: \[Hm] \[D] \[A] \[G] :/ \rep{2}}
    \endverse
  \fi

  \beginverse\memorize
    \[Hm] Meghívtál, hogy vízre \[D]lépjek, hol nélkü\[A]led elsüllye\[G]dek.
    \[Hm] Ebben megtalállak \[D]Téged, a mélység\[A]ben megt\_{a}rt hi\[G]tem.
  \endverse

  \beginchorus
    \[G] Nagy ne\[D]ved h\_{í}vom \[A]én \[A4] \[G] és felné\[D]zek a vizek fö\[A]lé.
    Ott l\_{á}tlak \[G]én. A lelkem \[D]benned m\_{e}gpi\[A]hen. Enyém v\[G]agy és \[A]én Ti\[Hm]ed.
  \endchorus

  \beginverse
    ^ A mélységnél nagyobb ke^gyelmed, mely elve^zet és tart en^gem.
    ^ Ha elbuknék és nagyon ^félnék, Te nem hagysz ^el és nem inogsz ^meg.
  \endverse

  \beginverse*
    Refrén
  \endverse

  \ifchorded
    \beginverse*
      {\nolyrics Közjáték: /: \[Hm] \[G] \[D] \[A] :/ \rep{2}}
    \endverse
  \fi

  \beginverse*
    /: \[Hm] Lélek add, hogy Benned \[G]teljesen megbízzak,
    A vízen \[D]bátran Veled járjak, és \[A]bárhová hívsz, menjek!
    \[Hm] Vigyél tovább, mint a \[G]lábam tudna menni,
    Taníts \[D]teljes hittel járni, jelen\[A]létedben élni! :/ \rep{3}
  \endverse

  \beginverse*
    Refrén \rep{2} \ifchorded{\nolyrics (Az elsőnek a végén \[D] -vel.)}\fi
  \endverse

  \beginverse*
    Enyém \[G]vagy és \[A]én Ti\[Hm]ed. Enyém \[G]vagy és \[A]én Ti\[D]ed.
  \endverse

\endsong


    \setcounter{songnum}{101}


    % \setcounter{songnum}{999}
    % \beginsong{Alleluja}
    % [
    %   by={Gocam ??? Varga Attila},
    %   li={Kék könyv / 230. [-2]}
    % ]

    %   \beginchorus
    %     /: \[Hm]Alleluja, \[A]alleluja, \[G]allelu\[F#]ja! :/ \rep{2}
    %   \endchorus

    %   \beginverse
    %     \[D]Ó, Urunk, mutassad \[A]meg nekünk, \[Hm]irgalmas szíve\[F#]det,
    %     \[D]És az üdvösséget \[A]add nekünk, \[Hm]add meg \[G]nékünk, Iste\[F#]nünk!
    %   \endverse

    %   \beginchorus
    %     /: \[Hm]Alleluja, \[A]alleluja, \[G]allelu\[F#]ja! :/ \rep{2}
    %   \endchorus

    % \endsong

    % \renewcommand{\thesongnum}{T\arabic{songnum}}
    % \setlength{\songnumwidth}{1.1cm}
    \renewcommand{\thesongnum}{T\arabic{songnum}}
\setlength{\songnumwidth}{1.1cm}

\beginsong{Ahol szeretet - !!}
[
  by={Taizé / Jacques Berthier}
]

  \beginverse*
    \[F]Ahol \[C]szere\[Dm]tet és \[A#]j\[D]ó\[G]ság, \[F]ahol \[C]szere\[Dm]tet, \[Gm]ott van \[C]Iste\[F]nünk.
  \endverse

\endsong

\beginsong{Áldott légy, Uram}
[
  by={Jacques Berthier}
]

  \beginverse*
    \[Dm]Áldott \[G]légy, U\[Dm]ram, szent \[A#]neved \[C]áldja \[F]lel\[A]kem!
    \[Dm]Áldott \[G]légy, U\[Dm]ram, mert \[A#]megvál\[C]tottál \[Dm]már.
  \endverse

\endsong

\beginsong{Bizakodjatok, jó az Úr - !!}
[
  by={Taizé / Jacques Berthier}
]

  \beginverse*
    \[D]Bizakodjatok, \[Hm]jó az Úr, \[D]jósága \[A]éltet,
    \[Em]Bizakodjatok, \[C]jó az Úr, \[Em]Allel\[A]u\[D]ja!
  \endverse

\endsong

\beginsong{Csak vándorolunk - !!}
[
  by={Taizé / Jacques Berthier}
]

  \beginverse*
    Csak \[Dm]vándorolunk az \[A#]éjben, mert \[C6]forrás vi\[Gm/A#]zére \[A4]vá\[A]gyunk.
    \[Dm]Szomjunk a \[C]fény a sö\[F]tét\[A]ben, szomjunk a \[A#]fény a sö\[A]tétben.
  \endverse

\endsong

\beginsong{Gyújts éjszakánkba fényt}
[
  by={Taizé / Jacques Berthier}
]

  \beginverse*
    \[H]Gyújts éjszakánkba \[Em]fényt, hadd égjen a
    Soha ki nem al\[D]vó \[G]tűz, a ki nem \[C]al\[G]vó \[D]tűz!
    Gyújts \[G]éjszakánkba \[C]fényt, hadd égjen a
    \[H]Soha ki nem \[Em]al\[Am6]vó \[H]tűz, a ki nem \[Em]al\[Am6]vó \[H]tűz!
  \endverse

\endsong

\beginsong{Irgalmas Istenünk}
[
  by={Jacques Berthier}
]

  \beginverse*
    \[Dm] Irgalmas \[A]Istenünk \[Dm]jósá\[C]gát \[F]mindö\[C]rökké \[Dm]é\[A]nek\[Dm]lem!
  \endverse

\endsong

\beginsong{Jézus életem}
[
  by={Taizé / Jacques Berthier}
]

  \beginverse*
    \[Dm]Jé\[C]zus \[F]életem, erőm, \[A#]bé\[C]kém,
    \[Dm]Jé\[C]zus \[F]társam, \[Dm]örö\[C]möm,
    Benned \[A#]bízom, \[A]Te vagy az \[Dm]Úr;
    Már \[C]nincs mit \[F]félnem, mert \[A#]bennem \[C]élsz,
    Már \[Am]nincs mit \[Dm]félnem, mert \[A#]ben\[C]nem \[F]élsz.
  \endverse

\endsong

\beginsong{Jézus, majd gondolj rám}
[
  by={Taizé / Jacques Berthier}
]

  \beginverse*
    \[D]Jézus, majd \[Em7]gondolj rám, \[A]ha a Te országod \[D]{\_{e}}ljön.
    \[Hm]Jézus, majd \[Em]gondolj rám, \[A]ha a Te országod \[D]eljön.
  \endverse

\endsong

\beginsong{Jó az Úrban bizakodni}
[
  by={Taizé / Jacques Berthier}
]

  \beginverse*
    \[Dm]Jó az Úrban \[A]bizakodni, {\[Dm]j}\[C]{\_{ó}} az \[F]Úr.
    \[Gm]Re\[C]mélj, és \[F]b{\_{í}}zz \[C]Ben\[Dm]ne, {\[A#]j}\[C]{\_{ó}} az \[Dm]Úr!
  \endverse

\endsong

\beginsong{Ne félj, ne aggódj}
[
  by={Jacques Berthier}
]

  \beginverse*
    \[Am]Ne félj, ne \[Dm7]aggódj, \[G]ne sírj, ne \[Cmaj7]bánkódj:
    \[F]Ha tiéd \[Dm6]Isten, \[E]tiéd már \[Am]minden.
    \[Am]Ne félj, ne \[Dm7]aggódj, \[G]ne sírj, ne \[Cmaj7]bánkódj:
    \[F]Elég \[Dm6]Ő \[E]né\[Am]ked.
  \endverse

\endsong

\beginsong{Az Úrra vár a szívünk}
[
  by={Jacques Berthier}
]

  \beginverse*
    Az \[Dm]Úrra \[C]vár a szí\[F]vünk,
    Ő\[Gm]benne \[Dm]minden \[A]örö\[Dm]münk.
  \endverse

  \ifchorded
    \beginverse*
      {\nolyrics ( Közjáték: \[Dm] \[A#] \[C]
      \[F] \[Dm] \[Em] \[A] )}
    \endverse
  \fi

\endsong

\beginsong{Várj és ne félj}
[
  by={Jacques Berthier}
]

  \beginverse*
    \[Em]Várj és ne \[C]félj, az \[Am6]Úr jön \[H7]már!
    \[Em]Várj \[D]és ne \[G]félj, hű \[Am7]szív\[H]vel \[Em]várj!
  \endverse

\endsong



    % \renewcommand{\thesongnum}{K\arabic{songnum}}
    \renewcommand{\thesongnum}{K\arabic{songnum}}

\beginsong{Hála Néked, Istenünk}[]

  \beginverse*\memorize
    \[G]Hála Néked, \[C]Istenünk, \[G]hála Néked \[D]\_\_\_\_\_ -ért!
    \[G]Hála Néked, \[C]Istenünk, mert \[G]ő fon\[D]tos ne\[G]künk! \[D] \[G]
  \endverse

  \beginverse*
    ^Alleluja, ^áldd Uram, ^alleluja, ^áldd Uram,
    ^Alleluja, ^áldd Uram, mert ^ő fon^tos ne^künk! ^ ^
  \endverse

  \beginverse*
    Pam-parararam \[D]pam \[G]pam param-param-pam-pam. \[G]
  \endverse

\endsong

\beginsong{Köszönjük Néked, Urunk}[]

  \beginverse*\memorize
    \[G]Köszönjük Néked, \[C]Urunk, _____-t, \[G] köszönjük Néked \[D]örökké,
    \[G]Ó, Urunk, mi \[C]nem hagyjuk el őt, és \[G]Veled \[D]dalol\[G]juk: \echo{Együtt, mindenki!}
  \endverse

  \beginverse*
    ^Yep ye-e-e-^ep ye-e-e-^ep ye-e-e-^ep oh oh oh!
    ^Yep ye-e-e-^ep ye-e-e ^yep oh ^yep oh ^yep \[D]yep \[G]yep!
  \endverse

  \beginverse*
    Pam-parararam \[D]pam \[G]pam param-param-pam-pam. \[G]
  \endverse

\endsong

\beginsong{Jó Atyánk, köszönjük Néked}[]

  \ifchorded
    \beginverse*
      {\nolyrics Előjáték: /: \[D] \[D4] \[D] \[D2] :/ \rep{2}}
    \endverse
  \fi

  \beginverse*\memorize
    \[D]Jó Atyánk, köszönjük Néked \[G]_____-t,
    Szívünkbe zártuk, \[D]hála \[D4]érte! \[D] \[D2]
  \endverse

  \beginverse*
    ^Áldj meg minket, Krisztusunk,
    A ^Lelked legyen közöttünk míg ^együtt ^vagyunk! ^ ^
  \endverse

  \ifchorded
    \beginverse*
      {\nolyrics \[H7]}
    \endverse
  \fi

  \beginverse*\memorize
    \[E]Ó, Urunk, mi \[A]egyet aka\[E]runk,
    Kik ebben a dalban \[H7]együtt vagyunk:
  \endverse

  \beginverse*
    ^Szereteted fénye, á^ldása érje ^_____ -t!
    Legyen a ^Lelked véle! \[E]Alle\[A]lu\[E]ja!
  \endverse

\endsong



    % \renewcommand{\thesongnum}{M\arabic{songnum}}
    % \setlength{\songnumwidth}{1.25cm}
    \beginsong{Uram, irgalmazz I.}
[
  by={Sillye Jenő},
  li={Kék könyv / 159.}
]

  \ifchorded
    \beginverse*
      {\nolyrics Előjáték:
        /: \[Dm] \[Gm] \[C7] \[F] :/ \rep{2}
        \[Dm] \[Gm] \[A7] \[Dm]
      }
    \endverse
  \fi

  \beginverse*
    \[Dm] Uram, irgal\[Gm]mazz! \[C7] Uram, irgal\[F]mazz!
    \[Dm] Krisztus, kegyel\[Gm]mezz! \[C7] Krisztus, kegyel\[F]mezz!
    \[Dm] Uram, irgal\[Gm]mazz! \[A7] Uram, irgal\[Dm]mazz!
  \endverse

  \ifchorded
    \beginverse*
      {\nolyrics Utójáték:
        \[Dm] \[Gm] \[C7] \[F]
        \[Dm] \[Gm] \[A7] \[Dm]
      }
    \endverse
  \fi

\endsong

\beginsong{Dicsőség I.}
[
  by={Sillye Jenő},
  li={Kék könyv / 159.}
]

  \beginverse*
    \[F] \[F]Dicsőség a \[Gm]magasságban \[C7] Isten\[F]nek,
    \[F]És a Földön \[Gm]békesség a \[C7]jószándékú \[Dm]embernek! \[Dm]
    \vspace{0.2cm}
    \[A#]Dicsőítünk \[C7]Téged, \[Gm] áldunk \[Dm]Téged,
    \[A#] Imádunk \[C7]Téged, \[Gm]magasztalunk \[Dm]Téged,
    \vspace{0.2cm}
    \[C7]Hálát adunk \[F]Neked \[Gm] nagy dicső\[F]ségedért,
    \[F] Urunk és \[Gm]Istenünk, \[C7] mennyei Ki\[F]rály,
    \[Gm]Mindenható \[C7]Atyais\[F]ten! \[Gm] \[C7] \[F]
    \vspace{0.2cm}
    \[A#]Urunk, Jézus \[C7]Krisztus, \[Gm]egyszülött Fi\[Dm]ú,
    \[A#] Urunk és \[C7]Istenünk, \[Gm] Isten \[Dm]Báránya,
    \vspace{0.2cm}
    Az \[C7]Atyának Fi\[F]a, Te elveszed a \[E]világ bűne\[A]it, \[A] \[Dm] irgalmazz \[Gm]nékünk; \[Gm]
    Te \[F]elveszed a \[E]világ bűne\[A]it, \[A] \[Dm] hallgasd meg könyörgésün\[Gm]ket.
    Te az \[F]Atya jobbján \[E]ülsz, \[E] irgalmazz \[A]nékünk!
    Mert \[D]egyedül Te vagy a Szent, Te vagy az \[A]Úr,
    Te vagy az \[F#m]egyetlen Föl\[E]ség, Jézus \[A]Krisztus,
    A \[F#m]Szentlélekkel \[Hm]együtt, az \[E]Atyaisten dicsőségé\[A]ben. \[G]
    \[A]{\_{A}}\[D]men.
  \endverse

\endsong

\beginsong{Szent vagy I.}
[
  li={Kék könyv / 192.}
]

  \ifchorded
    \beginverse*
      {\nolyrics Előjáték: \[F] \[Em] \[Am] \[Dm] \[G] \[G7] \[C]}
    \endverse
  \fi

  \beginverse\memorize
    Szent vagy, \[C]szent vagy, szent vagy, \[G7]sz{\_{e}}nt vagy
    Minden\[F]ségnek Ura, \[G]Istene \[C]
    \[C7] Dicső\[F]séged betölti a menny\[Em]et, és a föld\[Am]et
    Hozsan\[Dm]na \[G] a magas\[G7]ság\[C]ban!
  \endverse

  \beginverse
    Szent vagy, ^szent vagy, szent vagy, ^szent vagy
    Minden^ségnek Ura, ^Istene ^
    ^ \_{Ó} ^áldott, ki az Úr nevében ^eljön mihoz^zánk
    Hozsan^na ^ a magas^ság^ban!
  \endverse

  \ifchorded
    \beginverse*
      {\nolyrics Utójáték: \[F] \[Em] \[Am] \[Dm] \[G] \[G7] \[C]}
    \endverse
  \fi

\endsong

\beginsong{Isten báránya I.}
[
  by={Ferenczy Rudolf (Dax)},
  li={Kék könyv / 299.}
]

  \ifchorded
    \beginverse*
      {\nolyrics Előjáték: \[Em] \[Am] \[Ebdim] \[Em]}
    \endverse
  \fi

  \beginverse
    \[Em]Bűneinket, égi Bárány, \[Am]szent vérednek \[D7]drága árán
    \[G]Mind elveszed, \[Am]irgal\[H7]mazz ne\[Em]kem!
  \endverse

  \beginverse
    \[Em]Bűneinket, égi Bárány, \[Am]szent vérednek \[D7]drága árán
    \[G]Mind elveszed, \[Am]irgal\[H7]mazz ne\[Em]künk!
  \endverse

  \beginverse
    \[Em]Bűneinket, égi Bárány, \[Am]szent vérednek \[D7]drága árán
    \[G]Mind elveszed, \[Am]adj bé\[H7]két ne\[Em]künk!
  \endverse

  \ifchorded
    \beginverse*
      {\nolyrics Utójáték: \[Em] \[Am] \[Ebdim] \[Em]}
    \endverse
  \fi

\endsong


  \end{songs}
  \newpage
  \thispagestyle{empty}

  % To be continued...
  \vspace*{\fill}
  % \vfill
  
  Budaörs, 2020. november 01.
\end{document}
